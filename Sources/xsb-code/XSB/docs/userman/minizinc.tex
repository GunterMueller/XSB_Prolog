
\chapter[Interface to MiniZinc]{{\tt minizinc}: The XSB Interface to MiniZinc-based Constraint
  Solving}

\begin{center}
  {\Large {\bf By Michael Kifer}}
\end{center}



\section{Introduction}

MiniZinc is a uniform declarative constraint language that is understood by
most modern solvers for constraint and optimization problems. It comes bundled with a few such solvers, one of
which, \texttt{gecode}, is top-notch: very powerful and fast.  Other solvers,
including most of the newest ones, can be downloaded separately and
installed as plugins.

The MiniZinc language is described in
\url{https://www.minizinc.org/doc-2.2.3/en/index.html}; see, especially, the
tutorial. 
The XSB interface to MiniZinc comes with several sample problems from that
tutorial, which are found in \texttt{.../XSB/packages/minizinc/examples/}.
The file \texttt{.../XSB/packages/minizinc/examples.P} contains examples of
XSB invocations of those problems. After the installation, all examples can be run
by simply starting XSB and then
%% 
\begin{verbatim}
   | ?- [minizinc].   %% load the package
   | ?- [examples].   %% run all examples
\end{verbatim}
%% 

\section{Installation}

Some Linux distributions (e.g., Ubuntu) come with ready-made MiniZinc .deb
or .rpm packages. However, one must make sure that the command
\url{minizinc} is provided by those packages (some provide only the IDE).
Mac packages are also available.

In case a Linux or a Mac package is incomplete (or if one uses Windows),
MiniZinc can be downloaded from
\url{https://www.minizinc.org/software.html}. 

Once installed, make sure that the command \texttt{minizinc} is understood when typed in a
command window. If not, add \emph{folder-to-where-}\textbf{/bin/}minizinc is sitting to the
environment variable \texttt{PATH}.  In Windows, this is best done in
Control Panel; in Linux and Mac, add the command
%% 
\begin{verbatim}
    export PATH=$PATH:path-to-minizinc
\end{verbatim}
%% 
to \texttt{.bashrc} or an equivalent place. For instance,
%% 
\begin{verbatim}
    PATH=$PATH:$HOME/minizinc/MiniZincIDE-2.2.3-bundle-linux/bin
\end{verbatim}
%% 

\section{The API}

Constraint and optimization problems are specified using the MiniZinc
language in \emph{model files}, which have the suffix \texttt{.mzn}.
Such problems usually have a number of \emph{input variables}   and several
\emph{output} (or \emph{decision}) \emph{variables}.   
In principle, input values can be specified in the model file itself, but
this is generally not a good idea because typically one wants to solve
the same problem with different inputs.
For this reason, MiniZinc allows one to use models (that have no input)
with one or more \emph{input files}, which have the extension \texttt{.dzn}. 
In addition, XSB's API to MiniZinc lets one pass parameters in-line, as
part of a Prolog call.

\noindent
\emph{Important:} the MiniZinc model files (.mzn) must \textbf{not} have
\texttt{output} statements in them. Otherwise, errors and wrong answers
may result. (These output statements will be added automatically based on
output templates described below.)  

The API itself mainly consists of the following calls, which live in XSB's
module \texttt{minizinc}: 
%% 
\begin{itemize}
\item
  \texttt{solve(+MznF,+DatFs,+InPars,+Solver,+Solns,+OutTempl,-Rslt,-Xceptns)}: 
  The meaning of the arguments is as follows:
  %% 
  \begin{itemize}
  \item    \texttt{MznF}: should be bound to a path to the desired 
    model file (.mzn file) that contains a specification of a constraint or
    optimization problem. The path must be represented as a Prolog atom and
    can be absolute or relative to the current directory.
  \item \texttt{DatFs}: a list of paths (relative or absolute)  
    to the data files that describe all or some of the input parameters for
    the model in \texttt{MznF}. All paths must be atoms.
    If no data files are needed, just use the empty list \texttt{[]}. 
  \item \texttt{InPars}: a list of the \emph{in-line} input parameters to
    the model.
    These are supported for flexibility, to allow initialization of some or
    all input parameters directly in Prolog.
    Each parameter in \texttt{InPars} must have the form \emph{id} =
    \emph{value}, where \emph{id} must be the Id of an input variable used in
    the model file \texttt{MznF} and value must be a term understood by MiniZinc.
    Typically, such a term would be a number, an atom, or a function term.
    For instance,  \texttt{foo='[|1, 0|3, 4|8, 9|]'} would initialize the input
    variable \texttt{foo} with a 2-dimensional array of numbers;
    \texttt{'Item' = anon\_enum(8)}  would initialize the MiniZinc variable
    \texttt{Item}  of an enumerated type to a set
    \texttt{\{Item\_1,Item\_2,...,Item\_8\}}.  Note that \texttt{Item} must
    be quoted on the Prolog side, since otherwise it would be interpreted
    as a variable.
  \item \texttt{Solver}: the name of the solver to use. If this is a variable,
    the default solver \texttt{gecode} is used. 
  \item \texttt{Solns}: the number of solutions to show. 
    Could be \texttt{all} or a positive integer.
    Note: for optimization problems where a function is maximized or minimized,
    \texttt{Solns} \underline{must} be bound to 1. Otherwise, solvers would
    return also non-optimal solutions, and the desired optimal one may not
    be the first.
  \item \texttt{OutTempl}: the template showing how the output should look
    like. It has the form \emph{predname}($OutSpec_1$, ...,$OutSpec_n$)
    where \emph{predname} is the name of a predicate
    where the results will be stored (see below)
    and each $OutSpec$ can have one of these forms:
    %% 
    \begin{itemize}
    \item  An \emph{atom} representing an output (``decision'') variable
      defined in the model file \texttt{MznF} or an atom representing an
      arithmetic expression (in the syntax of MiniZinc, which is close to
      XSB) involving one or more decision variables.
      In the result, this atom will be replaced with the value of that
      variable or expression. Example: 'P*100'.
    \item A simple arithmetic expression involving decision variables,
      numbers, and +, $-$, *, /.  Example: p*100+9. The difference with the
      above is that the single quotes are omitted, for convenience.
      Note that if the MiniZinc variable name were P then it would have to
      be quoted---to protect it from Prolog: 'P'*100+9.
    \item  A term of the form \texttt{+}(\emph{atom}) or
      \texttt{str}(\emph{atom}).  In the result, this will be replaced with
      \emph{atom} verbatim. Note: if \texttt{atom} is not alphanumerical or
      if it starts with a capital letter, it must be quoted.

    \item A list of the form [$OutSpec_1$, ...,$OutSpec_n$]. Each
      \emph{OutSpec} in the list has the form
      described here.
    \item A term of the form $OutSpec_1$ \texttt{=} $OutSpec_2$. Each
      \emph{OutSpec} has the form described here.
    \end{itemize}
    %% 
    The template predicate cannot be \texttt{solve/8}, \texttt{solve\_flex/8},
    \texttt{show/1}, \texttt{delete/1},
    and the arity must be greater than 0.  
  \item \texttt{Rslt}: must be a term that matches the output
    template---typically just a variable.  Solutions to the constraint and
    optimization problems will be returned as bindings to variables in this
    term.  The term cannot match the predicates \texttt{solve/8},
    \texttt{solve\_flex/8}, \texttt{show/1}, or \texttt{delete/1} (e.g.,
    cannot be \texttt{solve(\_,\_,\_,\_,\_,\_,\_,\_)}), and the arity must
    be greater than 0 (i.e., the predicate cannot be \texttt{foo/0} and the
    like).

    \emph{Note}: solutions are also asserted in the predicate
    \texttt{minizinc:}\emph{SolutionPred}/\emph{Arity}, where
    \texttt{SolutionPred/Arity} is the predicate name and arity used in
    the aforementioned
    \texttt{OutTempl}.  Thus, solution predicates from different runs of
    MiniZinc are accumulated  in module \texttt{minizinc} and can be used
    at a later analysis stage. If any of these predicates are no longer
    needed, they can be emptied out with \texttt{retractall/1} or
    \texttt{abolish/1}. 
    For instance, \texttt{abolish(minizinc:somepred/5)}. 

  \item \texttt{Xceptns}: a list of exceptions returned by the solver.
    If the constraint/optimization problem was solved successfully, this
    variable will be bound to an empty list \texttt{[]}. Otherwise,
    each exception has the form
    \texttt{(reason=...,model\_file=...)}  and the
    following reasons might be returned:
    %% 
    \begin{itemize}
    \item  \texttt{unsatisfiable} --- the optimization problem is
      unsatisfiable.
    \item \texttt{unbounded}  --- the optimization problem has an unbounded
      objective function. For example, if the problem is to maximize the
      function and the function has no maximum (under the given
      constraints); or the problem is to minimize the function, and there
      is no minimum.
    \item \texttt{unsatisfiable\_or\_unbounded} --- one of the above.
    \item \texttt{unknown} --- could not find a solution within the limits
      (e.g., timeout).
    \item \texttt{error} --- search resulted in an error. 
    \end{itemize}
    %% 
  \end{itemize}
  %% 
  Note: exceptions are separate from other kinds of errors, such as a
  syntax errors in a model or data file or using a solver with a feature or
  option it does not support. Errors are explained later.
\item
  \texttt{solve\_flex(+MznF,+DatFs,+InPars,+Solver,+Solns,+OutTempl,-Rslt,-Xceptns)}: 
  The meaning of the parameters is the same as for \texttt{solve/8}.
  The difference is that if \texttt{InPars} or \texttt{OutTempl} 
  is non-ground, the call to
  MiniZinc is delayed until they both groun. If the top
  query finishes and \texttt{InPars} or \texttt{OutTempl} is still not ground, MiniZinc will
  \emph{not} be called at all.
  This is used in cases when it is hard to estimate when \texttt{InPars}
  or \texttt{OutTempl}
  may become ground, so calls to \texttt{solve\_flex} can be placed
  early. But, of course, one must ensure that \texttt{InPars}/\texttt{OutTempl}
  will get bound at some point.  
\end{itemize}
%% 

\paragraph{How does MiniZinc return complex structures back to Prolog?}
MiniZinc has a number of data structures that do not have direct
equivalence in Prolog, so they are mapped to Prolog terms when results are
returned as bindings for the \texttt{Rslt} variable. Here is the correspondence:
%% 
\begin{itemize}
\item  Numbers are passed back to Prolog as integers and floats.
\item  MiniZinc strings and identifiers are passed to Prolog as atoms.
\item  Arrays are returned as lists. Multi-dimensional arrays are flattened
  and passed as lists as well. For instance, a 2-dimensional array
  \texttt{[|1, 2|3, 4|5, 6|]} will be returned as \texttt{[1,2,3,4,5,6]}.
\item Sets are returned as terms of the form \texttt{\{elt1,elt2,elt3,...\}}. 
  For example, the set \texttt{\{a,b,c\}} comes back as the term
  \texttt{\{a,b,c\}}. Note that in Prolog this term is really
  \texttt{'\{\}'((a,b,c))}. Observe the double parentheses, which indicate
  that the functor symbol here is \texttt{\{\}/1}, not \texttt{\{\}/3}.
\item A MiniZinc range expressions of the form \texttt{N1..N2} is returned as
  \texttt{'..'(N1,N2)}. For instance, the range \texttt{3..17} 
  comes back to Prolog as \texttt{'..'(3,17)}. 
\end{itemize}
%% 

\paragraph{Errors vs. exceptions.}
If a model or data file contains a syntax error or a feature that the
chosen solver does not support, the \texttt{solve/8} and
\texttt{solve\_flex/8} calls will fail and a message will be printed to
standard output:
%% 
\begin{verbatim}
    +++ xsb2mzn: syntax or type errors found; details in ....some file...
\end{verbatim}
%% 
The user will then be able to find the details about the problem.

Note: errors are different from exceptions. If only exceptions are returned
(and no errors), the calls \texttt{solve/8} and \texttt{solve\_flex/8} will
succeed. In contrast, they will fail in case of errors.

\paragraph{Debugging API.}
For further development and bug reporting, the following calls are useful.
They are all 0-ary predicates that take no arguments; they all reside in the
XSB module \texttt{minizinc}. 
%% 
\begin{itemize}
\item  \texttt{keep\_tmpfiles}: The MiniZinc interface creates a number of
  temporary files, which are deleted, if MiniZinc finished normally and
  without an error. However, if a bug is suspected, it is desirable to
  preserve these files and send them to the developers. 
  This can be achieved by executing the \texttt{keep\_tmpfiles} predicate
  as a query, before the call to \texttt{solve/8}.
\item \texttt{show\_mzn\_cmd}: Executing this as a query will cause the
  \texttt{solve/8} predicate, described earlier, to print the shell command
  that was used to invoke MiniZinc in each call.
  This is useful if one suspects a bug in the API.
\item \texttt{dbg\_clear}: Executing this clears out the flags set by the
  above debugging calls. As a result, the temporary files will again be
  deleted after each invocation of MiniZinc and shell commands will not be
  shown.
\end{itemize}
%% 



%%% Local Variables: 
%%% mode: latex
%%% TeX-master: "manual2"
%%% End: 
