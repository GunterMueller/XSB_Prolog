\subsection{Tabling Aggregate Predicates}\label{tabling_aggregate_predicates}
%-------------------------------------------------------------------------
\index{aggregate predicates!tabling} \index{tabling!aggregate predicates}
%-------------------------------------------------------------------------

HiLog provides an elegant way to introduce aggregate operations into
XSB.  HiLog allows a user to define named (and parameterized) sets (or
bags).  For example, say we have a simple database-like predicate,
\verb|employee(Name,Dept,Sal)|, which contains a tuple for each
employee in our concern and contains the employee's name, department,
and salary.  From this predicate we can construct a set, or bag
really, that contains all the salaries of employees in the relation:
\begin{verbatim}
    :- hilog salaries.
    salaries(Sal) :- employee(_Name,_Dept,Sal).
\end{verbatim}
So \verb|salaries| is the name of a unary predicate that is true of
all salaries, or rather is the name of a {\em bag} of all salaries.
It is a bag since it may contain the same salary multiple times.
XSB provides a predicate \verb|bagSum| which can be used to
sum up the elements in a named bag.  So given the definition of the
HiLog predicate \verb|salaries/1| above, we can get the sum of all the
salaries with:
\begin{verbatim}
    :- bagSum(salaries,TotalSals).
\end{verbatim}
The first argument to \verb|bagSum| is the name of a bag, and the
second is bound to the sum of the elements in the bag.

We can also do a ``group by'' to get total salaries within departments
as follows.  We define a parameterized predicate, \verb|sals(Dept)|,
to be the bag of salaries of employees in department \verb|Dept|, as
follows:
\begin{verbatim}
    sals(Dept)(Sal) :- employee(_Name,Dept,Sal).
\end{verbatim}
This rule says that \verb|Sal| is in the bag named \verb|sals(Dept)|
if there is an employee with some name who works in department
\verb|Dept| and has salary \verb|Sal|.

Now with this definition, we can define a predicate,
\verb|deptPayroll/2|, that associates with each department the sum of
all the salaries of employees in that department:
\begin{verbatim}
    deptPayroll(Dept,Payroll) :- bagSum(sals(Dept),Payroll).
\end{verbatim}

XSB provides analogous aggregate operators, described below, to
compute the minimum, maximum, count, and average, of a bag,
respectively.  These predicates are all defined using a more basic
predicate \verb|bagReduce/4|.

\begin{description}

\ourrepeatstandarditem{bagReduce(?SetPred,?Arg,+Op,+Id)}{bagReduce/4}{HiLog,Tabling}
\ourstandarditem{filterReduce(?SetPred,?Arg,+Op,+Id)}{filterReduce/4}{Tabling}
%
{\tt SetPred} must be a HiLog set specification, i.e., a unary HiLog
predicate.  {\tt Op} must be a Hilog operation, i.e., a 3-ary HiLog
predicate that defines an associative operator.  The predicate must
define a binary function in which the first two arguments determine
the third.  {\tt Id} must be the identity of the operator.  {\tt
bagReduce} returns with {\tt Arg} bound to the ``reduce'' of the
elements of the bag determined by {\tt SetPred} under the operation
{\tt Op}.  I.e., {\tt Arg} becomes the result of applying the operator
to all the elements in the bag that unify with {\tt SetPred}.  See the
{\tt bagSum} operator below to see an example of {\tt bagReduce}'s
use.

{\tt filterReduce/4} acts as {\tt bagReduce/4} with two differences.
First, it does not depend on HiLog, so that {\tt filterReduce/4} will
be more robust especially when XSB's module system is used.  In
addition, {\tt filterReduce/4} aggregates solutions to {\tt Pred}
using a variance rather than unification.  An example of the use of
{\tt filterReduce/4} is given in Chapter \ref{chap:TablingOverview}.

\ourrepeatstandarditem{bagPO(?SetPred,?Arg,+Order)}{bagPO/3}{HiLog,Tabling}
\ourstandarditem{filterPO(?SetPred,?Arg,+Order)}{filterPO/3}{Tabling}
%
    {\tt SetPred} must be a HiLog set specification, i.e., a unary
    HiLog predicate.  {\tt Order} must be a binary Hilog relation that
    defines a partial order.  {\tt bagPO} returns nondeterministically
    with {\tt Arg} bound to the maximal elements, under {\tt Order}, of
    the bag {\tt SetPred}.  {\tt bagPO/3} can be used with {\tt Order}
    being subsumption to reduce a set of answers and keep only the most
    general answers.

    See the {\tt bagMax} operator below to see an example of {\tt
    bagPO}'s use.

{\tt filterPO/3} acts as {\tt bagPO/3} with the single difference that
it does not depend on HiLog, so that {\tt filterPO/3} will be more
robust especially when XSB's module system is used.

\ourstandarditem{filterPO(\#Pred,+Order)}{filterPO/2}{Tabling} 
%
{\tt filterPO(\#Pred,+Order)} succeds only for a solution $Pred\theta$
of {\tt Pred} for which there is no solution $Pred\eta$ to {\tt Pred}
such that {\tt Order($Pred\eta$,$Pred\theta$)}.

Example:

For the following program
     \begin{center}
     {\tt
     \begin{tabular}{l}
          :- table p/2.	\\
          b(1,2).       \\
          p(1,3).       \\
          b(1,1).       \\
\\
	  prefer(b(X,X),b(X,Y)):- X \== Y. 
     \end{tabular}
     }
     \end{center}
the query 
\begin{center}
{\tt ?- filterPO(b(X,Y)}
\end{center}
will succeed only with the binding {\em X = 1,Y = 1}.

\ourstandarditem{bagMax(?SetPred,?Arg)}{bagMax/2}{HiLog,Tabling}
%
    {\tt SetPred} must be a HiLog set specification, i.e., a unary
    HiLog predicate.  {\tt bagMax} returns with {\tt Arg} bound to the
    maximum element (under the Prolog term ordering) of the set {\tt
    SetPred}.  To use this predicate, it must be imported from aggregs,
    and you must give the following definitions in the main module {\tt
    usermod}:
\begin{verbatim}
:- hilog maximum.
maximum(X,Y,Z) :- X @< Y -> Z=Y ; Z=X.
\end{verbatim}
    (These decarations are necessary because of a current limitation in
    how HiLog predicates can be used.  This requirement will be lifted in
    a future release.)  With this definition, {\tt bagMax/2} can be (and
    is) defined as follows:
\begin{verbatim}
bagMax(Call,Var) :- bagReduce(Call,Var,maximum,_).
\end{verbatim}
    (Where variables are minimal in the term ordering.)

Another possible definition of {\tt bagMax/2} would be:
\begin{verbatim}
:- hilog lt.
lt(X,Y) :- X @< Y.

bagMax(Call,Var) :- bagPO(Call,Var,lt).
\end{verbatim}
This definition would work, but it is slightly less efficient than the
previous definition since it is known that {\tt bagMax} is
deterministic.

\ourstandarditem{bagMin(?SetPred,?Arg)}{bagMin/2}{HiLog,Tabling}
%
    {\tt SetPred} must be a HiLog set specification, i.e., a unary
    HiLog predicate.  {\tt bagMin} returns with {\tt Arg} bound to the
    minimum element (under the Prolog term ordering) of the set {\tt
    SetPred}.  To use this predicate, it must be imported from aggregs,
    and you must give the following definitions in the main module {\tt
    usermod}:
\begin{verbatim}
:- hilog minimum.  
minimum(X,Y,Z) :- X @< Y -> Z=X ; Z=Y.
\end{verbatim}
    (These decarations are necessary because of a current limitation in
    how HiLog predicates can be used.  This requirement will be lifted in
    a future release.)  With this definition, {\tt bagMin/2} can be (and
    is) defined as:
\begin{verbatim}
bagMin(Call,Var) :- bagReduce(Call,Var,minimum,zz(zz)).
\end{verbatim}
    (where structures are the largest elements in the term ordering.)

\ourstandarditem{bagSum(?SetPred,?Arg)}{bagSum/2}{HiLog,Tabling}
%
    {\tt SetPred} must be a HiLog set specification, i.e., a unary
    HiLog predicate.  {\tt bagSum} returns with {\tt Arg} bound to the sum
    of the elements of the set {\tt SetPred}.  To use this predicate, it
    must be imported from aggregs, and you must give the following
    definitions in the main module {\tt usermod}:
\begin{verbatim}
:- hilog sum.
sum(X,Y,Z) :- Z is X+Y.
\end{verbatim}
    (These decarations are necessary because of a current limitation in
    how HiLog predicates can be used.  This requirement will be lifted in
    a future release.)  With this definition, {\tt bagSum/2} can be (and
    is) defined as:
\begin{verbatim}
bagSum(Call,Var) :- bagReduce(Call,Var,sum,0).
\end{verbatim}

\ourstandarditem{bagCount(?SetPred,?Arg)}{bagCount/2}{HiLog,Tabling}
%
    {\tt SetPred} must be a HiLog set specification, i.e., a unary
    HiLog predicate.  {\tt bagCount} returns with {\tt Arg} bound to the
    count (i.e., number) of elements of the set {\tt SetPred}.  To use
    this predicate, it must be imported from aggregs, and you must give
    the following definitions in the main module {\tt usermod}:
\begin{verbatim}
:- hilog successor.
successor(X,_Y,Z) :- Z is X+1.
\end{verbatim}
    (These decarations are necessary because of a current limitation in
    how HiLog predicates can be used.  This requirement will be lifted in
    a future release.)  With this definition, {\tt bagCount/2} can be (and
    is) defined as:
\begin{verbatim}
bagCount(Call,Var) :- bagReduce(Call,Var,successor,0).
\end{verbatim}

\ourstandarditem{bagAvg(?SetPred,?Arg)}{bagAvg/2}{HiLog,Tabling}
%
    {\tt SetPred} must be a HiLog set specification, i.e., a unary
    HiLog predicate.  {\tt bagAvg} returns with {\tt Arg} bound to the
    average (i.e., mean) of elements of the set {\tt SetPred}.  To use
    this predicate, it must be imported from aggregs, and you must give
    the following definitions in the main module {\tt usermod}:
\begin{verbatim}
:- hilog sumcount.
sumcount([S|C],X,[S1|C1]) :- S1 is S+X, C1 is C+1.
\end{verbatim}
    (These decarations are necessary because of a current limitation in
    how HiLog predicates can be used.  This requirement will be lifted in
    a future release.)  With this definition, {\tt bagAvg/2} can be (and
    is) defined as:
\begin{verbatim}
bagAvg(Call,Avg) :- 
    bagReduce(Call,[Sum|Count],sumcount,[0|0]),
    Avg is Sum/Count.
\end{verbatim}

\end{description}
