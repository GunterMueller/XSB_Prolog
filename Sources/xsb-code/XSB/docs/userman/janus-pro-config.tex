
This chapter first describes how to configure \janusplg , followed by
introductory examples.  Next is a more precise description of its
functions, its current limitations followed by applications and
further examples.

\section{Configuration and Loading}

XSB's configure/make processes will also confgure and make \janus{}
for Linux, Mac and Windows.  On these platforms, \janusplg{} has been
tested using versions 3.5 -- 3.11 of Python and its
libraries.\footnote{The
\janusplg{} configuration script was written by Michael Kifer.  It has
been tested on Windows 10; on Ubuntu, Fedora and CentOs Linuxes; and
on MacOs.}  The \janusplg configuration script can be found in {\tt
XSB/packages/janus}.  \janusplg configuration also can be run
separately from the main configuration script if desired.

\subsection{Installing \janusplg{} to work with Anaconda}

Installing \janusplg under Anaconda has been tested under a few
versions of Anaconda.  Anaconda has the advantage that it includes all
of the Python development extensions needed for \janusplg.  To make
use of them, simply activate your conda environment, (re)configure and
(re)make XSB as described in the manual.  However, there are numerours
configurations for Anaconda, and some configurations include separate
compilers and library files than are found in the main operating
system.  As a result it is {\em {\bf critical}} that the XSB
configured for a particular anaconda environment is {\em always} used
within that environment, and {\em only} within that environment.  If
XSB is to be used outside of a given anaconda environment or in a
different environmen<t a separate copy of XSB should be downloaded and
configured.

\subsection{Configuring \janusplg{} for {\tt venv} or without a virtual environment} \label{sec:xsbpy-linux}

If you use {\tt venv} as your environment manager or if you are
configuring a system installation of {\tt xsbpy} for several users,
configuration may be slightly more complicated than using Anaconda.
In principle, when XSB is configured on Linux or the Mac, \janusplg{}
will also be configured so that it can be used like any other 
module.  However, for this configuration to work several pieces of
software much be in place.

\begin{itemize}
\item {\tt git} and {\tt gcc} (or some other compiler) must be present
  and usable by the user performing the configuration.  Most of these
  tools need to be installed on Linux or macOS to configure and
  compile XSB, so they will not usually be an issue.

\item A version of C-Python (the usual implementation of Python) 3.x
  with development extensions must be present and usable.\footnote{In
    particular {\tt libpython} and {\tt Python.h} are needed from the
    development version.}  Unfortunately, Python development
  extensions are not contained in a {\tt venv} environment (unlike
  with Anaconda) so they will need to be installed.\footnote{{\tt
      venv} is not designed to support applications like \janusplg{}
    that embed Python in another process.  When a virtual environment
    is created using {\tt venv}, the virtual environment contains a
    Python executable, but not {\tt libpython} or {\tt Python.h} which
    must be linked or included from system directories.}

  \item \janusplg{} configuration uses a Python package called {\tt
    find\_libpython}, and will try to do a {\tt pip} installation if
    it is not present.  {\tt pip} will install this package under the
    {\tt venv} or user's {\tt site-packages} directory; otherwise a
    system {\tt site-packages} directory will be used.\footnote{Some
    Linuxes like Ubuntu do not include {\tt pip} in their default
    distribution(!), in which case the Linux package for {\tt pip}
    must be installed.}
\end{itemize}


%\subsubsection{Installing {\tt xsbpy} to work with a system Python}

\subsubsection{Troubleshooting}

%If you have {\tt sudo} privileges and want to make {\tt xsbpy} work
%with a system version of Python there are a few more steps needed than
%when working with Anaconda.  First, your system needs a development
%version of Python~\footnote{If several versions of Python 3.x are
%  installed, the most recent will be used by the configure script.}
%The Python development extensions are needed because they contain
%Python as a shared object ({\tt .so}) file along with Python header
%({\tt .h}) files.  While some version of Python-3 is included in
%almost any version of Linux, additional Linux package installations
%either may be needed for these development extensions or to install a
%more recent version of Python.
%  
%  As an example, suppose version 3.8 of Python is desired.  If it is
%  not present on the platform, it may be installed with its
%  development extensions as:

%  {\tt sudo apt-get install python3.8-dev}
%
%\noindent
%  in Ubuntu; or
%
%  {\tt sudo dnf install python3.10-dev}
%
%\noindent
%  in Fedora (newer versions) or CentOs.
%
%%Otherwise, if Python3.8 has been installed but the development version
%%is not present the development extensions ({\tt libpython3.8.so} and
%{\tt Python.h}) can be installed as:
%
%{\tt apt-get install libpython3.8-dev}.

If there are difficulties in configuring \janusplg , it is likely that
one of the above requirements is not met.  However, if these
requirements are met, but the configuration does not work properly, it
is best to check the file {\tt packages/janus/xsb2py\_connect\_defs.h}
to check whether its definitions are correct ({\tt
  xsb2py\_connect\_defs.h} is created when Janus is
configured).\footnote{If you can determine that the required software
is present and that there is a problem with the configuration please
report the problem at {\tt https://sourceforge.net/p/xsb/bugs}.}

The definitions in {\tt xsb2py\_connect\_defs.h} are used in {\tt
  get\_compiler\_options/2} in {\tt init\_janus.P} to properly compile
{\tt xsbpy} and link in {\tt libpython3x.so}.  These final steps are
done automatically when \janusplg{} is consulted into XSB via a normal
consult or {\tt ensure\_loaded/[1,2]}.  As part of this process, in
{\tt init\_janus.P} the main C code for  \janusplg{} is compiled using
gcc compiler options that are appropriate for the version of {\tt
  libpython} used.\footnote{See {\tt get\_compiler\_options/2} in {\tt
  init\_janus.P}.}  When the  \janusplg{} module is consulted into XSB,
the {\tt libpython} shared object file is loaded dynamically along
with the \janus{} shared object file.

%If installing {\tt xsbpy} on Ubuntu, the first step is to ensure that
%the proper Python libraries and header files are present.  If using
%Python version 3.7 this command is:
%
%{\tt sudo apt-get install libpython3.7-dev}
%
%For {\tt xsbpy} to work properly in Linux:
%\begin{itemize}
%\item The {\tt xsbpy} code must be compiled and dynamically loaded into XSB.
%\item A version of {\tt libpython.so} also needs to be dynamically
%  loaded into XSB, and configured with the proper C library paths.
%\item The proper paths must be set for the Python Standard Library.
%\item Paths to {\tt xsbpy} and to {\tt xsbpy/apps} must be set up to
 % access both XSB and Python modules.
%\end{itemize}

%\begin{verbatim}
%-I <path to XSB's emu directory>  
%-I/usr/include/python3.7m  -I/usr/include/python3.7  
%-lpython3.7m -L/usr/lib/python3.7 
%-Wl -rpath=/usr/lib/python3.7 -lpython3.7m -L/usr/lib/python3.7 
%-Wl -rpath=/usr/lib/python3.7
%\end{verbatim}

\noindent
%In the above compilation options, {\tt xsbpy} initialization
%configures the path to the XSB {\tt emu} directory but all other paths
%are currently hard-coded.

\subsection{Installing a development version of Python under  macOS (if Anaconda is not used)}

Most of the discussion in Section~\ref{sec:xsbpy-linux} pertains
directly to macOS.  As with Linux, installation of \janus{} under
Anaconda will be easiest for most users, and it is worthwhile noting
that whether or not Anaconda is used both XSB and \janus{} can be
compiled using either {\tt clang} or {\tt gcc} on both macOS or Linux.
%When XSB is configured on the Mac, the configuration script also tries
%to configure {\tt xsbpy} so that it can be loaded and used like any
%other module.  However for this process to work correctly additional
%software must be present.
%
%\subsubsection{A Development Version of Python}
%
However, if XSB is not configured using an Anaconda version of Python
a development version of Python 3.x needed, but is not included by
default in {\tt macOS}, so that code needed by {\tt xsbpy} like {\tt
  Python.h} or {\tt libpython<xxx>.dylib} needs to be installed.  This
software is not always contained in {\tt xtools} either.

Fortunately, there are several installation tools available for the
Mac.  One of the most popular is {\tt brew}.  Once {\tt brew} itself
is installed it will be used to make two further installations.
First, type:

{\tt brew install autoconf automake libtool}

\noindent
which is a prerequisite for a development version of Python to be
installed.  Next, type 

{\tt brew install python@3.8}

\noindent
to install the Python itself (other version numbers of Python can be used)
and save the output of this installation.  Brew installs this Python in a
directory that for many users is not on the execution path determined by
the \texttt{PATH} environment variable.  Therefore the directory
\emph{Python-installation-folder}\texttt{/bin/} will need to be added to
the user's execution path (to the \texttt{PATH} variable), as instructed in
the brew installation output that appears after installing Python.
This can be done, for instance, by executing
%% 
\begin{verbatim}
echo 'export PATH="/usr/local/opt/python@3.8/bin:$PATH"' >> ~/.zshrc
echo 'export LDFLAGS="-L/usr/local/opt/python@3.8/lib"' >> ~/.zshrc
\end{verbatim}
%% 
if the development version of
Python is installed in \texttt{/usr/local/opt/python@3.8}. This
addition is important, because XSB's configuration checks the user's
execution paths as one way to find software.

%\subsubsection{A C Compiler and Loader}
%
%Second -- naturally -- there must be a C compiler installed on your
%Mac.  C compilers can be installed on the Mac through Xcode or Apple's
%Command Line Tools.  (A user needn't be familiar with how a compiler
%works -- the compiler just needs to be installed).  Two compilers are
%available on macOS, Clang and GCC; XSB and {\tt xsbpy} have been
%tested using both compilers.
%
%\subsubsection{The Configuration Itself}
%
%Once a C compiler has been installed; and a development version of
%Python has been installed and added to the user's execution paths,
%simply run the {\tt configure} and {\tt makexsb} scripts in {\tt
%  \$XSB\_ROOT/XSB/build} and {\tt xsbpy} will be built automatically.
%
%As you can see, the entire process is simplicity itself!
%
\subsection{Configuring {\tt xsbpy} under Windows}

To run the {\tt xsbpy} interface under Windows 10, you must first
download and install Python as described above.

You must set a windows environment variable (local or global) named
PYTHON\_LIBRARY to the name of the Python DLL, which is part of the
Python system.  That DLL can be found by using the python script:
{\tt find\_libpython} executed on a Python command line.  To run this
script, you will probably have to install it, using the command
{\tt pip install find-libpython}, start Python, and then type:
%% 
\begin{verbatim}
   from find_libpython import find_libpython
   find_libpython()
\end{verbatim}
%% 
This will print out the name of
the file containing the necessary DLL.  Simply define the environment
variable PYTHON\_LIBRARY to be this .dll file.  (If you don't have
administrative permission on your computer, you can define it using
``Edit the environment variables for your account'' from that
lower-left windows command lookup box.)

To see if the interface is installed correctly and is functional, it
is suggested (just below) that you run: {\tt bash} test.sh in
directory {\tt <XSB>/packages/xsbpy/test/}.  While this can be made to
work when cygwin is installed, under windows, it may be easier to run
{\tt xsb} in that directory and do:
\begin{verbatim}
| ?- [testSuite], testSuite.
\end{verbatim}
This should load the Python interface and run a few tests, printing
out results.  If it succeeds without errors and without printing
failed test notifications, the interface is available for use.

It is possible that XSB will try to recompile the interface, if the
file times are inconsistent.  In this case, be sure that the file time
of {\tt xsbpym.xwam} is later than that of {\tt xsbpym.H} in {\tt
  <XSB>/packages/xsbpy/} and try again.
This cab be done in the Windows command window like this (assuming
the current directory is the root of your XSB installation):
%% 
\begin{verbatim}
  copy /b packages\xsbpy\xsbpym.xwam +,,
\end{verbatim}
%% 
(You may alternatively simply
rename {\tt xsbpym.H} to another name.)

\subsection{Testing whether the {\tt xsbpy} configuration was successful}

No matter what platform {\tt xsbpy} is installed on, to test out
whether {\tt xsbpy} has been loaded and is working properly, simply
change the working directory to {\tt packages/xsbpy/test} and execute the
command

\begin{verbatim}
  bash test.sh
\end{verbatim}


