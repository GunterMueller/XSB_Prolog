\newcommand{\janus}{{\tt janus}}
\newcommand{\janusplg}{\texttt{janus-plg}}
\newcommand{\januspy}{\texttt{janus-py}}
\newcommand{\janusxsb}{\texttt{janus\_xsb}}
\newcommand{\jnsversion}{Version 2.0}

\begin{center}
\chapter[Janus: Prolog calling Python]{Janus: Calling Python from Prolog} \label{chap:januspy}
\end{center}

\vspace*{-.30in} 
\begin{center}
{\Large {\bf  \jnsversion}}
\end{center}

\begin{center}
  {\Large {\bf By Theresa Swift, Muthukumar Suresh, Carl Andersen}}
\end{center}

\noindent
%{\large {\bf {\em This chapter documents beta-level software.}}}

%Both XSB and Python are written in C, which makes possible a robust
%and efficient interface.  
%

The new \janus{} package provides an easy and highly efficient way for
Prolog to call Python3 functions and methods, and vice versa.  \janus{}
is originally based on the packages {\tt xsbpy} and {\tt px}
\cite{SwiA23,AndS23} and has undergone a major rewrite with expanded
functionality~\footnote{The {\tt xsbpy} and {\tt px} packages were
partly funded by BBN Technologies.} in close collaboration with the
SWI Prolog and Ciao Prolog teams~\footnote{An earlier version of \janus{}
is also supported by Arriba Prolog. The documentation in this
chapter and the next is based mostly on previous {\tt xsbpy} and {\tt
  px} documentation but also includes some material originally written
for SWI.} This chapter describes Prolog calling Python, while Chapter
~\ref{chap:janus-py}  describes Python calling Prolog.

\janus{} leverages the fact that the reference C-Python is written in
C (as are most Prologs), so that Prolog and Python can be readily
loaded into the same process. The core interface routines are also
written almost entirely in C, so the interface is very efficient
(hundreds of thousands to {\em millions} of round-trip calls between
Prolog and Python can be made per second) and it is hoped -- very
robust within its known restrictions.  In addition, due to the dynamic
typing of both Prolog and Python, a simple bi-translation allows
complex Python data structures to be mapped to and from Prolog terms.
All of this this makes using Python from Prolog as simple as
consulting \janus{} from XSB and calling Python.  Calling Prolog from
Python is as simple: just import \janus{} from Prolog and start making
calls.


This chapter first describes how to configure \janusplg , followed by
introductory examples.  Next is a more precise description of its
functions, its current limitations followed by applications and
further examples.

\section{Configuration and Loading}

XSB's configure/make processes will also confgure and make \janus{}
for Linux, Mac and Windows.  On these platforms, \janusplg{} has been
tested using versions 3.5 -- 3.11 of Python and its
libraries.\footnote{The
\janusplg{} configuration script was written by Michael Kifer.  It has
been tested on Windows 10; on Ubuntu, Fedora and CentOs Linuxes; and
on MacOs.}  The \janusplg configuration script can be found in {\tt
XSB/packages/janus}.  \janusplg configuration also can be run
separately from the main configuration script if desired.

\subsection{Installing \janusplg{} to work with Anaconda}

Installing \janusplg under Anaconda has been tested under a few
versions of Anaconda.  Anaconda has the advantage that it includes all
of the Python development extensions needed for \janusplg.  To make
use of them, simply activate your conda environment, (re)configure and
(re)make XSB as described in the manual.  However, there are numerours
configurations for Anaconda, and some configurations include separate
compilers and library files than are found in the main operating
system.  As a result it is {\em {\bf critical}} that the XSB
configured for a particular anaconda environment is {\em always} used
within that environment, and {\em only} within that environment.  If
XSB is to be used outside of a given anaconda environment or in a
different environmen<t a separate copy of XSB should be downloaded and
configured.

\subsection{Configuring \janusplg{} for {\tt venv} or without a virtual environment} \label{sec:xsbpy-linux}

If you use {\tt venv} as your environment manager or if you are
configuring a system installation of {\tt xsbpy} for several users,
configuration may be slightly more complicated than using Anaconda.
In principle, when XSB is configured on Linux or the Mac, \janusplg{}
will also be configured so that it can be used like any other 
module.  However, for this configuration to work several pieces of
software much be in place.

\begin{itemize}
\item {\tt git} and {\tt gcc} (or some other compiler) must be present
  and usable by the user performing the configuration.  Most of these
  tools need to be installed on Linux or macOS to configure and
  compile XSB, so they will not usually be an issue.

\item A version of C-Python (the usual implementation of Python) 3.x
  with development extensions must be present and usable.\footnote{In
    particular {\tt libpython} and {\tt Python.h} are needed from the
    development version.}  Unfortunately, Python development
  extensions are not contained in a {\tt venv} environment (unlike
  with Anaconda) so they will need to be installed.\footnote{{\tt
      venv} is not designed to support applications like \janusplg{}
    that embed Python in another process.  When a virtual environment
    is created using {\tt venv}, the virtual environment contains a
    Python executable, but not {\tt libpython} or {\tt Python.h} which
    must be linked or included from system directories.}

  \item \janusplg{} configuration uses a Python package called {\tt
    find\_libpython}, and will try to do a {\tt pip} installation if
    it is not present.  {\tt pip} will install this package under the
    {\tt venv} or user's {\tt site-packages} directory; otherwise a
    system {\tt site-packages} directory will be used.\footnote{Some
    Linuxes like Ubuntu do not include {\tt pip} in their default
    distribution(!), in which case the Linux package for {\tt pip}
    must be installed.}
\end{itemize}


%\subsubsection{Installing {\tt xsbpy} to work with a system Python}

\subsubsection{Troubleshooting}

%If you have {\tt sudo} privileges and want to make {\tt xsbpy} work
%with a system version of Python there are a few more steps needed than
%when working with Anaconda.  First, your system needs a development
%version of Python~\footnote{If several versions of Python 3.x are
%  installed, the most recent will be used by the configure script.}
%The Python development extensions are needed because they contain
%Python as a shared object ({\tt .so}) file along with Python header
%({\tt .h}) files.  While some version of Python-3 is included in
%almost any version of Linux, additional Linux package installations
%either may be needed for these development extensions or to install a
%more recent version of Python.
%  
%  As an example, suppose version 3.8 of Python is desired.  If it is
%  not present on the platform, it may be installed with its
%  development extensions as:

%  {\tt sudo apt-get install python3.8-dev}
%
%\noindent
%  in Ubuntu; or
%
%  {\tt sudo dnf install python3.10-dev}
%
%\noindent
%  in Fedora (newer versions) or CentOs.
%
%%Otherwise, if Python3.8 has been installed but the development version
%%is not present the development extensions ({\tt libpython3.8.so} and
%{\tt Python.h}) can be installed as:
%
%{\tt apt-get install libpython3.8-dev}.

If there are difficulties in configuring \janusplg , it is likely that
one of the above requirements is not met.  However, if these
requirements are met, but the configuration does not work properly, it
is best to check the file {\tt packages/janus/xsb2py\_connect\_defs.h}
to check whether its definitions are correct ({\tt
  xsb2py\_connect\_defs.h} is created when Janus is
configured).\footnote{If you can determine that the required software
is present and that there is a problem with the configuration please
report the problem at {\tt https://sourceforge.net/p/xsb/bugs}.}

The definitions in {\tt xsb2py\_connect\_defs.h} are used in {\tt
  get\_compiler\_options/2} in {\tt init\_janus.P} to properly compile
{\tt xsbpy} and link in {\tt libpython3x.so}.  These final steps are
done automatically when \janusplg{} is consulted into XSB via a normal
consult or {\tt ensure\_loaded/[1,2]}.  As part of this process, in
{\tt init\_janus.P} the main C code for  \janusplg{} is compiled using
gcc compiler options that are appropriate for the version of {\tt
  libpython} used.\footnote{See {\tt get\_compiler\_options/2} in {\tt
  init\_janus.P}.}  When the  \janusplg{} module is consulted into XSB,
the {\tt libpython} shared object file is loaded dynamically along
with the \janus{} shared object file.

%If installing {\tt xsbpy} on Ubuntu, the first step is to ensure that
%the proper Python libraries and header files are present.  If using
%Python version 3.7 this command is:
%
%{\tt sudo apt-get install libpython3.7-dev}
%
%For {\tt xsbpy} to work properly in Linux:
%\begin{itemize}
%\item The {\tt xsbpy} code must be compiled and dynamically loaded into XSB.
%\item A version of {\tt libpython.so} also needs to be dynamically
%  loaded into XSB, and configured with the proper C library paths.
%\item The proper paths must be set for the Python Standard Library.
%\item Paths to {\tt xsbpy} and to {\tt xsbpy/apps} must be set up to
 % access both XSB and Python modules.
%\end{itemize}

%\begin{verbatim}
%-I <path to XSB's emu directory>  
%-I/usr/include/python3.7m  -I/usr/include/python3.7  
%-lpython3.7m -L/usr/lib/python3.7 
%-Wl -rpath=/usr/lib/python3.7 -lpython3.7m -L/usr/lib/python3.7 
%-Wl -rpath=/usr/lib/python3.7
%\end{verbatim}

\noindent
%In the above compilation options, {\tt xsbpy} initialization
%configures the path to the XSB {\tt emu} directory but all other paths
%are currently hard-coded.

\subsection{Installing a development version of Python under  macOS (if Anaconda is not used)}

Most of the discussion in Section~\ref{sec:xsbpy-linux} pertains
directly to macOS.  As with Linux, installation of \janus{} under
Anaconda will be easiest for most users, and it is worthwhile noting
that whether or not Anaconda is used both XSB and \janus{} can be
compiled using either {\tt clang} or {\tt gcc} on both macOS or Linux.
%When XSB is configured on the Mac, the configuration script also tries
%to configure {\tt xsbpy} so that it can be loaded and used like any
%other module.  However for this process to work correctly additional
%software must be present.
%
%\subsubsection{A Development Version of Python}
%
However, if XSB is not configured using an Anaconda version of Python
a development version of Python 3.x needed, but is not included by
default in {\tt macOS}, so that code needed by {\tt xsbpy} like {\tt
  Python.h} or {\tt libpython<xxx>.dylib} needs to be installed.  This
software is not always contained in {\tt xtools} either.

Fortunately, there are several installation tools available for the
Mac.  One of the most popular is {\tt brew}.  Once {\tt brew} itself
is installed it will be used to make two further installations.
First, type:

{\tt brew install autoconf automake libtool}

\noindent
which is a prerequisite for a development version of Python to be
installed.  Next, type 

{\tt brew install python@3.8}

\noindent
to install the Python itself (other version numbers of Python can be used)
and save the output of this installation.  Brew installs this Python in a
directory that for many users is not on the execution path determined by
the \texttt{PATH} environment variable.  Therefore the directory
\emph{Python-installation-folder}\texttt{/bin/} will need to be added to
the user's execution path (to the \texttt{PATH} variable), as instructed in
the brew installation output that appears after installing Python.
This can be done, for instance, by executing
%% 
\begin{verbatim}
echo 'export PATH="/usr/local/opt/python@3.8/bin:$PATH"' >> ~/.zshrc
echo 'export LDFLAGS="-L/usr/local/opt/python@3.8/lib"' >> ~/.zshrc
\end{verbatim}
%% 
if the development version of
Python is installed in \texttt{/usr/local/opt/python@3.8}. This
addition is important, because XSB's configuration checks the user's
execution paths as one way to find software.

%\subsubsection{A C Compiler and Loader}
%
%Second -- naturally -- there must be a C compiler installed on your
%Mac.  C compilers can be installed on the Mac through Xcode or Apple's
%Command Line Tools.  (A user needn't be familiar with how a compiler
%works -- the compiler just needs to be installed).  Two compilers are
%available on macOS, Clang and GCC; XSB and {\tt xsbpy} have been
%tested using both compilers.
%
%\subsubsection{The Configuration Itself}
%
%Once a C compiler has been installed; and a development version of
%Python has been installed and added to the user's execution paths,
%simply run the {\tt configure} and {\tt makexsb} scripts in {\tt
%  \$XSB\_ROOT/XSB/build} and {\tt xsbpy} will be built automatically.
%
%As you can see, the entire process is simplicity itself!
%
\subsection{Configuring {\tt xsbpy} under Windows}

To run the {\tt xsbpy} interface under Windows 10, you must first
download and install Python as described above.

You must set a windows environment variable (local or global) named
PYTHON\_LIBRARY to the name of the Python DLL, which is part of the
Python system.  That DLL can be found by using the python script:
{\tt find\_libpython} executed on a Python command line.  To run this
script, you will probably have to install it, using the command
{\tt pip install find-libpython}, start Python, and then type:
%% 
\begin{verbatim}
   from find_libpython import find_libpython
   find_libpython()
\end{verbatim}
%% 
This will print out the name of
the file containing the necessary DLL.  Simply define the environment
variable PYTHON\_LIBRARY to be this .dll file.  (If you don't have
administrative permission on your computer, you can define it using
``Edit the environment variables for your account'' from that
lower-left windows command lookup box.)

To see if the interface is installed correctly and is functional, it
is suggested (just below) that you run: {\tt bash} test.sh in
directory {\tt <XSB>/packages/xsbpy/test/}.  While this can be made to
work when cygwin is installed, under windows, it may be easier to run
{\tt xsb} in that directory and do:
\begin{verbatim}
| ?- [testSuite], testSuite.
\end{verbatim}
This should load the Python interface and run a few tests, printing
out results.  If it succeeds without errors and without printing
failed test notifications, the interface is available for use.

It is possible that XSB will try to recompile the interface, if the
file times are inconsistent.  In this case, be sure that the file time
of {\tt xsbpym.xwam} is later than that of {\tt xsbpym.H} in {\tt
  <XSB>/packages/xsbpy/} and try again.
This cab be done in the Windows command window like this (assuming
the current directory is the root of your XSB installation):
%% 
\begin{verbatim}
  copy /b packages\xsbpy\xsbpym.xwam +,,
\end{verbatim}
%% 
(You may alternatively simply
rename {\tt xsbpym.H} to another name.)

\subsection{Testing whether the {\tt xsbpy} configuration was successful}

No matter what platform {\tt xsbpy} is installed on, to test out
whether {\tt xsbpy} has been loaded and is working properly, simply
change the working directory to {\tt packages/xsbpy/test} and execute the
command

\begin{verbatim}
  bash test.sh
\end{verbatim}




\section{Introductory Examples}

We introduce some of the core functionality of \janusplg{} via a
series of simple examples.  As background, when \janus{} is loaded,
Python is also loaded and initialized within the XSB process, the core
Prolog modules of \janus{} are loaded into XSB, and paths to \janus{}
and its sub-directories are added both to Prolog and to Python (the
latter paths are added by modifying Python's {\tt sys.path}). Later,
XSB calls Python, Python will search for modules and packages in the
same manner as if they were stand-alone.

\begin{example} \rm {\bf Calling a Python Function (I)} \label{ex:janus-json}
  %

  {{\em The translation of JSON through \janusplg{} in this example
      works well, but for most purposes we recommend using XSB's native
      JSON interface described in Chapter \ref{chap:json} of this
      manual.  }}

Suppose \janus{} has been loaded by the command {\tt ?- [janus].} and
consider the call:

\begin{verbatim}
py_func(json,prolog_loads('{"name":"Demo term"
                     "created": {"day":null,
                                 "month":"December",
                                 "year":2007  },
                     "confirmed":true,
                     "members":[1,2,3]}'
\end{verbatim}
%'{"name": "Bob", "languages": ["English","French","GERMAN"]}'),
%       Ret)
 
\noindent 
loads the Python {\tt json} module
%in {\tt  packages/jason/starters}
and then calls the Python function

{\tt json.loads()}

\noindent
with the above JSON string as its argument.
%\verb|'{"name": "Bob", "languages": ["English","French","GERMAN"]}'|
%as the argument.
In Python the atom is parsed and converted into a Python dictionary
whose syntax is very close to that of the JSON.  Next, \janus{}
translates this dictionary to a Prolog term that can be pretty
printed as:
\begin{verbatim}
{name:'Demo term'),
 created:{day:@(none),
          month:'December',
          year:2007},
 confirmed:@(true),
 members:[1,2,3]}
\end{verbatim}
%%
The syntactic flexibility of Prolog allows the Python dictionary to be
represented as a logical term whose syntax is very close to that of
both a Python dictionary and a JSON object.  We call a term that maps
to a Python dictionary a ``Janus dictionary term''.  Such a term is
simply a Prolog term whose outer functor is \verb|'{}'|/1, whose
argument is a comma list, where the elements of the comma list are
attribute-value pairs (sometimes called key-value pairs) using the
predicate {\tt :/2}, where the attributes and values themselves may
contain nested dictionaries, lists, tuples or sets in accordance with
the restrictions of Python dictionaries.\footnote{In Python,
attributes may only contain mixtures of integers, strings and tuples.}
\end{example}

\begin{example} \rm {\bf Calling a Python Function (II): Where to Maintain Python Objects?} \label{jns-examp:glue}

\noindent
A slightly more complex call to Python is to load a JSON object from a
file as opposed to a string.  For this, we may use the Python function
{\tt

  {\tt json.load(Stream)}}

\noindent
which loads a JSON string from a file into a Python dictionary which
can then be translated to Prolog.  A small problem arises in that the
input to {\tt json.load()} is a Python input stream (sometimes called
a file pointer {\tt fp} is Python docs).  Python input streams are of
course different than Prolog input streams.  This can be handled in
several ways.

\begin{itemize}
  \item {\bf Maintain The Stream Reference in Python}

A straightforward solution to the problem is to write a small amount of
glue code in Python as follows.

\begin{verbatim}   
def prolog_load(File):
    with open(File) as fileptr:
        return(json.load(fileptr))
\end{verbatim}
\noindent
If this code were kept in the file {\tt jns\_json.py} the call
\begin{verbatim}
py_func(jns_json,prolog_load('sample.json'),Json)
\end{verbatim}
would unify {\tt Json} with a \janus{} dictionary term as in the last
example.

\item {\bf Maintain The Stream Reference in Prolog}
  %
\janus{} also allows the user to obtain any Python object reference in
Prolog.  The goal
\begin{verbatim}   
py_func(builtins,open('sample.json',r),Stream),
py_func(json,load(Stream),Json),
py_dot(Stream,close(),Return).
\end{verbatim}
makes three calls to Python: one to obtain the Python stream as a
Python object reference, a second to parse the JSON object from the
file and load it into Prolog, and a third to close the stream.  Note
that closing a stream requires a method to be applied to a stream
object rather than a function call, so the \janus{} predicate {\tt
  py\_dot/[3,4]} is called instead of {\tt py\_func/[3,4]}.
\end{itemize}

Each of these approaches has advantages.  Maintaining the object in
Prolog requires no glue code, but does require explicitly opening and
closing the file.  Either works well from the viewpoint of
performance: the \janus{} interface is so fast that making one call
vs. three calls to Python make no measurable difference in most cases
(see Section~\ref{janus-pro:performance}).  In general, the decision of
whether to maintain object references in Python or Prolog is largely a
matter of taste.
\end{example}

\begin{example} \rm {\bf Calling a Python Function (III): Keyword Arguments} \label{ex:jns-kewords}
  
\noindent
Python library functions often make heavy use of keyword arguments.
These are easily handled by {\tt py\_func/[3,4]} along with other \janus{}
functions such as {\tt py\_dot/[3,4]} and {\tt py\_iter/[3,4]}.  Suppose we
want to call the Python function

{\tt json.dumps(Dict,indent=2)}

\noindent
where {\tt Dict} is a Python dictionary that is to be written out as a
JSON string.  This can easily be called via

{\tt py\_func(json,dumps(PlgDict,indent=2),Ret)}

\noindent
where {\tt PlgDict} is a Prolog dictionary that is to be converted to
the Python dictionary {\tt Dict}. Note that {\tt py\_func/[3,4]}
handles keyword arguments using the same syntax as Python: positional
arguments must occur first in a call followed by 0, 1 or more keyword
arguments.
\end{example}

%The previous examples have sketched an approach that can efficiently
%call virtually any Python function or method,\footnote{{\tt xsbpy}
%  does not currently support Python's binary types.}  although it
%might require a small amount of glue code.  However, Python methods can
%also be called directly.

The \janus{} predicate {\tt py\_dot/3} was briefly introduced in
Example~\ref{jns-examp:glue}.  Let's take a closer look at it.

\begin{example} \rm {\bf Calling a Python Method} \label{jns-examp:method}

\noindent
Consider the following simple Python class:

\begin{verbatim}
class Person:
  def __init__(self, name, age, ice_cream=None):
    self.name = name
    self.age = age
    if favorite_ice_cream is None:
      favorite_ice_cream = 'chocolate'
    self.favorite_ice_cream = favorite_ice_cream

  def hello(self,mytype):
    return("Hello my name is " + self.name + " and I'm a " + mytype)
\end{verbatim}

\noindent
The call

\begin{verbatim}
    py_func('Person','Person'(john,35),Obj),
\end{verbatim}
\noindent
creates a new instance of the {\tt Person} class, and returns a
reference to this instance that can later be used to call a method.
We refer to this reference abstractly as $<obj>$ as the form of a
Python object reference can differ between Prologs that support \janus .
Using this reference, the goal

  {\tt  py\_dot($<obj>$,hello(programmer),Ret2).}

\noindent
returns the Prolog atom:

{\tt 'Hello my name is john and I'm a programmer'}

Note that unlike {\tt py\_func/[3,4]} which requires a module as its
first argument, the module is not needed in {\tt py\_dot/[3,4]} as the
module is implicit in the object reference.
%\noindent
%Although Python methods, like Python functions, can include keyword
%arguments, {\tt xsbpy} does not support keyword arguments in {\tt
%%  py\_dot/4} because Version 3.9.4 of the Python C API does not permit
%this.
\end{example}

\begin{example} \rm {\bf Examining a Python Object} \label{jns-examp:exam-object}

\noindent
Example \ref{jns-examp:method} showed how to create a Python object,
pass it back to Prolog and apply a method to it.  Suppose we create
another {\tt Person} instance:

\begin{verbatim}
    py_func('Person','Person'(bob,34),Obj),
\end{verbatim}
\noindent
and later want to find out all attributes of {\tt bob} both explicitly
assigned, and default.  This is easily done by {\tt
  janus:obj\_dict/2}:

{\tt obj\_dict(Obj ,ObjDict ).}

\noindent
returns
\begin{verbatim}
    ObjD = {name:bob,age:34,favorite_ice_cream:chocolate}
\end{verbatim}

There are times when using the dictionary associated with a class is
either not possible or not appropriate.  For instance, not all Python
classes have {\tt \_\_dict\_\_} methods defined for them, or only a
single attribute of an object might be required.  In these cases, {\tt
  py\_dot/4} can be used:

{\tt    py\_dot('Person',$<obj>$,favorite\_ice\_cream,I)}

\noindent
returns {\tt I = chocolate}.

\noindent
Summarizing from the previous two examples, {\tt py\_dot/[3,4]} can be
used in two ways.  If the second argument of a call to {\tt py\_dot/4}
is a Prolog structure, the structure is interpreted as a method.  In
this case, a Python method is applied to the object, and its return is
unified with the last argument of {\tt py\_dot/4}. If the second
argument is a Prolog atom, it is interpreted as attribute of the
object.  In this case, the attribute is accessed and returned to
Prolog.  Note that the functionality of {\tt py\_dot/4} is overloaded
in direct analogy to the functionality of the {\tt '.'} connector in
Python.
\end{example}

\begin{example} \rm {\bf Eager and Lazy Returns} \label{jns-examp:lazy-ret}
\noindent
  Prolog can either ``lazily'' backtrack through solutions to a goal
  $G$ or ``eagerly'' return all solutions to $G$ as a list via {\tt
    findall/3} or similar predicates.  In an analogous manner, Python
  can either 1) return a list or set of returns via a mechanism such
  as comprehension; or 2) return solutions one at a time through the
  {\tt yield} statement or similar framework¡.  \janus{} provides full
  flexibility in handling both lazy and eager returns.

  Consider a file {\tt range.py} that contains the following functions:
\begin{verbatim}
  def demo_yield(): 
    for i in range(1000000):
        yield i

  def demo_comp():
    return [i for i in range(1000000)]
\end{verbatim}
%
To improve performance, many Python libraries, such as SpaCy and the
RDF-HDT interface to Wikidata, use {\tt yield} to return generators
rather than lists or other data structures.  In this example, we
consider accessing each of {\tt yield} and large Python data
structures both eagerly and lazily.  These two cases are abstractly
represented by {\tt demo\_yield()} and {\tt demo\_comp()}.

We first address the case of {\tt demo\_yield()}.  The goal 

{\tt py\_func(range,demo\_yield(),YieldObj).}

\noindent
returns a reference to a Python object.  This is because at the Python
level {\tt demo\_yield()} actually returns a generator object, which
is not one of the object types bi-translated by \janus{}
(Section~\ref{sec:jns-bi-translation}).  \janus{} provides several ways
to handle the return of an object through which we would like to
iterate.  For instance to goal:

{\tt py\_iter(RangeObject,Return).}

Will backtrack through all solutions yielded by the generator object
produced by {\tt demo\_yield()}.  An even simpler approach is to
directly call:

{\tt py\_iter(range,demo\_yield(),Return).}

\noindent
which will also backtrack through all the returns generated by {\tt
  demo\_yield()}.

%\noindent
%Another approach is to call 
%
%{\tt py\_next(Obj,Return).}
%
%\noindent
%using the generator object ({\tt Obj}) returned by the {\tt
%  py\_func/3} call above.  {\tt py\_next/2} will also backtrack
%through the returns of {\tt demo\_yield()}.  {\tt py\_next} will only
%`vbacktrack through objects that are iterators and so won't backtrack
%through the returns of {\tt demo\_comp()}.

In contrast to backtracking through the returns, \janus{} can also access
{\em eagerly} access lazy returns from Python.  The call

{\tt py\_func(range,demo\_yield(),[do\_iter(true)]).}

\noindent
immediately returns as a list all 1,000,000 elements produced by the
generator.  Although eagerly returning the output of generator is easy
to do in \janus , it should be cautiously used, as Python generators can
iterate though a large number of solutions.\footnote{This is
particularly true when using the RDFLIB-HDT package.}  For instance
the interface to a database cursor may be a function that uses a {\tt
  yield} statement.

%\noindent
Beyond the {\tt yield} statement, \janus{} can also return data
structures lazily from Python.  Consider the function {\tt
  demo\_comp()} introduced above, which returns a list of 1,000,000
elements. The call

{\tt py\_iter(range,demo\_comp(),Return).}

\noindent
will backtrack through the million elements of the list returned by
{\tt demo\_comp()} one by one.\footnote{The code for {\tt py\_iter()}
accesses the iterator for the Python return if one is available, in
which case the sequence is returned lazily from Python to Prolog. If
no iterator is available, the entire data structure is returned.}

Alternately, if an object reference to the list of 1,000,000 is
desired, a user could also call 

{\tt py\_func(range,demo\_comp(),CompObj,[py\_object(true)]).}

which will unify {\tt CompObj} with a Python object reference.  This
can be followed by:

{\tt py\_iter(CompObj,Return)}

which will produce the same answers as the goal {\tt
  py\_iter(range,demo\_comp(),Return)} mentioned above.

\end{example}

To summarize Example~\ref{jns-examp:lazy-ret}:

\begin{itemize}
  \item {\tt py\_iter/3} can backtrack through Python returns based on
    the {\tt yield} statement, an idiom that is commonly used in
    Python libraries.  In fact, if {\tt py\_iter/3} is given {\em any}
    reference to a Python object belonging to an iterator class, it
    can backtrack through all returns produced by iteration.
  \item In a similar manner, {\tt py\_iter/3} can also backtrack
    through a large data structure, $D_{large}$ such as one returned
    by a Python library such as Elasticsearch.  Alternately, a
    reference to $D_{large}$ can be returned via the {\tt
      py\_object(true)} option and used by {\tt py\_iter/2} to return
    its elements one by one.
    \item {\tt py\_func/4} (or {\tt py\_dot/4}) with the {\tt
      do\_iter(true)} option can be used to return at once all answers
      from a call to Python that uses a {\tt yield} statement, or that
      returns any other iterator object.
\end{itemize}

\begin{example} {\bf py\_call/[2,3]} \rm \label{jns-examp:pycall}

\noindent
    {\tt py\_call/[2,3]} provides an alternate syntax for
    {\tt py\_dot/[3,4]} and {\tt py\_func/[3,4]}.  Rather than calling

{\tt py\_func(Module,Function,Return)}

\noindent
one may equivalently call 

{\tt py\_call(Module:Function,Return)}

\noindent
and rather than calling 

{\tt py\_dot(Object,Function,Return)}

\noindent
one may equivalently call 

{\tt py\_call(Object:Function,Return)}

\noindent
These equivalences also hold when options are provided for a call.

The syntax of {\tt py\_func/[3,4]} and {\tt py\_dot/[3,4]} is arguably
slightly more ``Pythonic'' than {\tt py\_call/[2,3]}.  Python
distinguishes between calling a function and applying a method or
obtaining an attribute and this distinction is maintained when using
{\tt py\_func/[3,4]} and {\tt py\_dot/[3,4]}.  On the other hand, {\tt
  py\_call/[2,3]} is arguably slightly more ``Prologic'', since it
treats module qualification in the same manner as with Prolog goals,
and does not require the user to distinguish between Python methods
and functions.

There is no deep difference between the two approaches: they are
merely alternate syntaxes.  In XSB, {\tt py\_call/[2,3]} is defined in
terms of {\tt py\_func/[3,4]} and {\tt py\_dot/[3,4]}; in SWI it is
the reverse.  Which form to use is a matter of taste.
\end{example}

%\subsection{Summary of Examples}

%A great deal of Python functionality is directly available via {\tt
%  py\_func/[3,4]} and {\tt py\_dot/[3,4]} and {\tt py\_iter/3}.  In
%our experience so far, many Python libraries can be called directly
%from Prolog and will ``just work'' immediately.  Cases where glue code
%is needed or useful include the following.
%
%\begin{itemize}
%\item In a case like Example \ref{xsbpy-examp:glue} where a Python
%  method or function like {\tt json.load()} requires a Python resource
%  as input there are two choices.  First, a small amount of code might
%  be useful to, say, open a file and perform an operation.  As an
%  alternative, the file can be opened, the file pointer passed back to
%  XSB as an object reference, and the function called directly from
%%  XSB using the object reference.
%
%\item Suppose a class with several attributes is defined as a subclass
%  of, say a string type.  The default behavior of {\tt janus} is to
%  simply pass back such objects as strings, rather than as object
%  references.  An example of this in fact occurs in the sample
%  interface {\tt packages/janus/starters/jns\_rdflib} (see
%  Section~\ref{secLxp-rdflib}).  In the {\tt rdflib} package {\tt
%    rdflib.Literal} objects are in fact subclasses of a string type.
%  These {\tt rdflib.Literal} objects have additional attributes
%  representing language tags and data type designations that are
%%  critical for RDF I/O from Prolog.  One way of handling this is seen
%  in {\tt jns\_rdflib.py}, where slightly more elaborate glue code is
%  needed to marshal an object's attributes as elements of a tuple, and
%  pass them back along with the object.  An alternative is to call
%  rdflib using the {\tt [py\_object(true)]} option, which will return
%  each row as an object that can be examined by {\tt janus} calls.
%\end{itemize}

%With those disclaimers in mind, all glue code that we have needed to
%write so far has been simple and straightforward.

\section{Bi-translation between Prolog Terms and Python Data Structures} \label{sec:jns-bi-translation}

\janus{} takes advantage of a C-level bi-translation of a large
portion of Prolog terms and Python data structures: i.e., Python
lists, tuples, dictionaries, sets and other data structures are
translated to their Prolog term forms, and Prolog terms of restricted
syntax are translated to lists, tuples, dictionaries, sets and so on.
Bi-translation is recursive in that any of these data structures can
be nested in any other data structures (subject to limitations on
mutables in Python).
     
Due to syntactic similarities between Prolog terms and Python data
structures, the Prolog term forms are easy to translate and use -- and
sometimes appear syntactically identical.

\index{jns\_struct}
\index{jns\_term}
\index{term form (of a structure}
%
As terminology, when a Python data structure $D$, say a dictionary, is
translated into a Prolog term $T$, $T$ is sometimes called the {\em
  term form} of $D$.  The type representing any Python structure that
can be translated to Prolog is called {\em jns\_struct} while the type
representing a Prolog term that can be translated into a Python data
structure is called a {\em jns\_term}

\subsection{The Bi-translation Specification} \label{sec-bi-translation}

Bi-translation between Prolog and Python can be described from the
viewpoint of Python types as follows:

\begin{itemize}
       \item {\em Numeric Types}: Python integers and floats are
         bi-translated to Prolog integers and floats.  Python complex
         numbers are not (yet) translated, and integers are only
         supported for integers between XSB's minimum and maximum
         integer~\footnote{These integers can be obtained by querying
           {\tt current\_prolog\_flag/2}.}
         \begin{itemize}
           \item {\em Boolean Types} in Python are translated to the
             special Prolog structures {\tt @(true)} and {\tt
               @(false)}.
         \end{itemize}
       \item {\em String Types}: Python string types are bi-translated
         to Prolog atoms.  This translation assumes UTF-8 encoding on
         both sides.

         Note that a Python string can be enclosed in either double
         quotes (\verb|''|) or single quotes (\verb|'|).  In
         translating from Python to Prolog, the outer enclosure is
         ignored, so Python {\tt "'Hello'"} is translated to the
         Prolog {\tt '\textbackslash{}'Hello\textbackslash{}'{}'},
         while the Python {\tt '"Goodby"'} is translated to the Prolog
         {\tt '"Goodby"'}.
       \item {\em Sequence Types}:
         \begin{itemize}
           \item Python lists are bi-translated as Prolog lists and
             the two forms are syntactically identical.  The maximum
             size of lists in both XSB and Python is limited only by
             the memory available.
           \item A Python tuple of arity {\tt N} is bi-translated with
             a compound Prolog term \verb|-/N| (i.e., the functor is a
             hyphen).  The maximum size of tuples in XSB is $2^{16}$.
%             \item Python ranges are not (yet) translated (i.e., they
%               are returned as terms with functor {\tt pyObj/1}).
         \end{itemize}
       \item {\em Mapping Types}: The translation of Python
         dictionaries takes advantage of the syntax of braces, which
         is supported by any Prolog that supports DCGs. The term form
         of a dictionary is;

         \verb|{ DictList} |

         where {\tt DictList} is a comma list of {\tt ':'/2} terms
         that use input notation.

         {\tt Key:Value}

         {\tt Key} and {\tt Value} are the translations of any Python
         data structures that are both allowable as a dictionary key
         or value, and supported by \janus .  For instance, {\tt
           Value} can be (the term form of) a list, a set, a tuple or
         another dictionary.  For instance the Python dictionary:

         {\tt \{'K1':[1,2,3], 'k2':(4,5,6)]\}}

         has a nearly identical term form

         {\tt \{'K1':[1,2,3], k2:-(4,5,6)]\}}

         \item {\em Set Types}: A Python set {\em S} is translated to
         the term form

         {\tt py\_set(SetList)}

         where {\em SetList} is the list containing exactly the
         translated elements of $S$.  Due to Python's implementation
         of sets, there is no guarantee that the order of elements
         will be the same in $S$ and $SetList$.
       \item {\em None Types.} The Python keyword {\tt None} is
         translated to the Prolog term {\tt @(none)}. 
       \item {\em Binary Types:} are not yet supported.  There are no
         current plans to support this type.
     \item Any Python object {\tt Obj} that is a non-primitive type,
       or of a type that is not translated to a specific Prolog term
       as indicated above, is translated to the Prolog object
       reference, which can be passed back to Python for an object
       call or other purposes.  In XSB, object references have the
       form {\tt pyObj(Obj)}, but this form is system dependent, and
       will differ in other Prologs that support \janus{} such as
       SWI.
\end{itemize}

%Additionally, a user with a minimal knowledge of C can change parts of
%the syntax used in Prolog term forms.  The outer functors {\tt
%  pyDict}, {\tt py_set} and {\tt pyObj} and the constant {\tt None} can
%all be redefined my modifying the file {\tt xsbpy\_defs.h} in the {\tt
%  xsbpy} directory.

\section{Usage}

\begin{description}

\indourrepeatjnsitem{py\_func(+Module,+Function,?Return)}{py\_func/3}
\indourjnsitem{py\_func(+Module,+Function,?Return,+Options)}{py\_func/4}
%
 Ensures that the Python module {\tt Module} is loaded, and calls {\tt
   Module.Function} unifying the return of {\tt Function} with {\tt
   Return}.  As in Python, the arguments of {\tt Function} may contain
 keywords but positional arguments must occur before keywords.  For
 example the goal

\begin{verbatim}
py_func(jns_rdflib,rdflib_write_file(Triples,'out.ttl',format=turtle),Ret).
\end{verbatim}

calls the Python function {\tt jns\_rdflib.rdflib\_write\_file()} to
write {\tt Triples}, a list of triples in Prolog format, to the file
{\tt new\_sample.ttl} using the {\tt turtle} format.
%This format is
%specified as a keyword argument to {\tt rdflib\_write\_file()} in the
%third argument of {\tt py\_func/4}.

In general, {\tt Module} must be the name of a Python module or path
represented as a Prolog atom.  Python built-in functions can be called
using the ``pseudo-module'' {\tt builtins}, for instance

{\tt  py\_func(builtins, float('+1E6'),F).}

\noindent
produces the expected result:

{\tt F = 1000000.0}

\noindent
If {\tt Module} has not already been
  loaded, it will be automatically loaded during the call.  Python modules are
  searched for in the paths maintained in Python's {\tt sys.path} list
  and these Python paths can be queried from XSB via {\tt
    py\_lib\_dir/1} and modified via {\tt py\_add\_lib\_dir/1}.
     
{\tt Function} is the invocation of a Python function in {\tt Module},
where {\tt Function} is a compound Prolog structure in which arguments
with the outer functor {\tt =/2} are treated as keyword arguments.
%Optional keyword arguments are passed
%in the third argument as lists of {\tt Key = Value} terms; if no such
%arguments are needed, {\tt Kwargs} can be an empty list -- or {\tt
%  py\_func/3} may be used.  Finally the return value from {\tt
%  Function} is unified with {\tt Return}.

Currently supported options are:
\begin{itemize}
  \item {\tt py\_object(true)} This option returns most Python data
    structures as object references, so that attributes of the data
    structures can be queried if needed.  The only data returned {\em
      not} as an object reference are
    \begin{itemize}
    \item Objects of {\tt boolean} type
    \item Objects of {\tt none} type
    \item Objects of exactly the class {\tt long}, {\tt float} or {\tt
      string}.  Objects that are proper subclasses of these types are
      returned as object references.
    \end{itemize}
  \item {\tt do\_iter(true)} has close to the opposite effect of {\tt
    py\_object(true)}.  For a Python object {\tt Obj} that is not a
    list, tuple, set or dictionary, but that implements the iterator
    protocol, all iterated values of {\tt Obj} are returned in a list.
    Such objects include Python generators as well as instances of any
    Python class that supports the iterator protocol, such as {\tt
      range()}.
\end{itemize}
%The only Prolog bjeopttion currently allowed is {\tt sizecheck(trbue)},
%which traverses the tPython data structure to determine its size before
%returning the data structure to XSB.  (See
%Section~\ref{sec:xsbpy-memory} for details of this option.)

{\bf Error Cases}

\bi
\item {\tt py\_func/4} is called with an uninstantiated option list
\bi
\item {\tt instantiation\_error}
\ei
\item The option list {\tt py\_func/4} contains an improper element,
  or combination of elements.
  \bi
\item {\tt domain\_error}
  \ei
\item {\tt Module} is not a Prolog atom:
\bi
\item {\tt type\_error}
\ei
\item {\tt Module} cannot be found in the current Python search paths:
\bi
\item {\tt existence\_error}
\ei
\item {\tt Function} is not a callable term
\bi
\item {\tt type\_error}
  \ei
\item {\tt Function} does not correspond to a Python function in {\tt Module}
\bi
\item {\tt existence\_error}
  \ei
\item When translating an argument of function:
  \begin{itemize}
  \item A set ({\tt py\_set/1}) term has an argument that is not a list
    \begin{itemize}
    \item {\tt type\_error}
    \end{itemize}
  \item The list in a set term ({\tt py\_set/1} contains a non-hashable term
    \begin{itemize}
    \item {\tt type\_error}
    \end{itemize}
  \item A dictionary ({\tt {}/1}) term has an argument that is not a
      comma-list
    \begin{itemize}
    \item {\tt type\_error}
    \end{itemize}
  \item An element of a dictionary comma-list is not of the form {\tt
    :/2} or the structure contains a non-hashable key (first argument)
    \begin{itemize}
    \item {\tt type\_error}
    \end{itemize}
  \item An argument of {\tt Function} is otherwise non-translatable to Python
    \begin{itemize}
    \item {\tt misc\_error}
    \end{itemize}
  \end{itemize}
\end{itemize}
%
  In addition, errors thrown by Python are caught by XSB and re-thrown
  as {\tt misc\_error} errors.

\indourrepeatjnsitem{py\_dot(+ObjRef,+MethAttr,?Ret,+Prolog\_Opts)}{py\_dot/4}
\indourjnsitem{py\_dot(+ObjRef,+MethAttr,?Ret)}{py\_dot/3}
%
Applies a method to {\tt ObjRef} or obtains an attribute value for
\texttt{ObjRef}.  As with {\tt py\_func/[3,4]}, {\tt ObjRef} is a
Python object reference in term form or a Python module.  A Python
object reference may be returned by various calls, such as
initializing an instance of a class: \footnote{In XSB this is a term
of the form {\tt pyObj(Ref)} or {\tt pyIter(Ref)} where {\tt Ref} is a
Prolog atom depicting the actual reference to a Python object.
However, other implementations of \janus{} use other conventions.}
%{\tt py\_dot/4} acts in one of two ways:
\begin{itemize}
\item If {\tt MethAttr} is a Prolog compound term corresponding to a
  Python method for {\tt ObjRef}, the method is called and its return
  unified with {\tt Ret}.

%  Unfortunately, limitations in the Python C API version 3.9.4 lead to
%  two limitations in calling Python methods from XSB through C.
%  First, keyword arguments cannot be used when calling Python method
%  as they can for Python functions.  Second, {\tt MethAttr} must have
%  3 or fewer arguments.\footnote{The reason for this is that to
%    execute Python n-ary methods from the C-API a variadic function
%     call must be made, and variadic function {\em calls} cannot be
%    constructed dynamically in C (or at any rate, I don't know how to
%    do this).}
%
\item If {\tt MethAttr} is a Prolog atom corresponding to the name of
  an attribute of {\tt ObjRef}, the attribute value (for \texttt{ObjRef})
  is accessed and unified
  with {\tt Ret}.
\end{itemize}

Both the Prolog options ({\tt Prolog\_Opts}) and the handling of Python
paths is as with {\tt py\_func/[3,4]}.

{\bf Error Cases}
\bi
\item {\tt py\_dot/4} is called with an uninstantiated option list
\bi
\item {\tt instantiation\_error}
\ei
\item The option list {\tt py\_dot/4} contains an improper element,
  or combination of elements.
  \bi
\item {\tt domain\_error}
  \ei
\item {\tt Obj} is not a Prolog atom or Python object reference
\bi
\item {\tt type\_error}
\ei
`\item {\tt MethAttr} is not a callable term or atom.
\bi
\item {\tt type\_error}
  \ei
\item {\tt MethAttr} does not correspond to a Python method or
  attribute for {\tt PyObj}
  \bi
\item {\tt misc\_error}
  \ei
\item If an error occurs when translating an argument of {\tt
  MethAttr} to Python the actions are as described for {\tt
  py\_func/[3,4]}.  
  \end{itemize}
%\end{itemize}
%
In addition, errors thrown by Python are
  caught by XSB and re-thrown as {\tt misc\_error} errors.

  %\bi
%\item {\tt ObMod} is neither a Python Object reference, nor a module string
%\bi
%\item {\tt type\_error}
%\ei
%\item {\tt ObjRef} is not a Python object reference in Prolog term form:
`%\bi
%\item {\tt misc\_error}
%\ei
%\item {\tt MethAttr} is neither a Prolog compound term nor a Prolog atom:
%\bi
%\item {\tt misc\_error}
%\item {\tt MethAttr} has arity greater than 3.
%\bi
%\item {\tt misc\_error}
%  \ei
%  \ei \ei 

\indourjnsitem{py\_iter(+ModObj,+FuncMethAttr,Ret)}{py\_iter/3}
%\indourjnsitem{py\_iter(+ObjRef)}{py\_iter/4}
%
{\tt py\_iter/2} takes as input to its first argument either a module
in which the function {\tt FuncMethAttr} will be called; a Python
object reference to which either the method {\tt FuncMethAttr} will be
applied or the attribute {\tt FuncMethAttr} will be accessed.  Just as
with {\tt py\_func/[3,4]} and {\tt py\_dot/[3,4]} the arguments of
{\tt FuncMethAttr} may contain keywords, but positional arguments must
occur before keywords.  However, if the Python function, method or
attribute returns an iterator object {\tt Obj}, the iterator for {\tt
  Obj} will be accessed and values of the iterator will be returned
via backtracking (cf. Example~\ref{jns-examp:lazy-ret}.

If the size of a return from Python is expected to be very large, say
over 1MB or so the use of {\tt py\_iter()} is recommended.

{\bf Error Cases}

Error cases are similar to {\tt py\_func/[3,4]} if {\tt ModObj} is a
module, and to {\tt py\_obj} if {\tt ModObj} is a Python object
reference.

%\indourjnsitem{py\_next(+ObjRef)}{py\_next/1}

\indourrepeatjnsitem{py\_call(+Form,-Ret,+Opts)}{py\_call/3}
\indourjnsitem{py\_call(+Form,Ret)}{py\_call/2}
%
{\tt py\_call/[2,3]} is alternate syntax for {\tt py\_func/[3,4]} and
{\tt py\_dot/[3,4]}.

{\tt py\_call(Mod:Func,Ret,Opts)}

\noindent
emulates {\tt py\_func(Mod,Func,Ret,Opts)}, while

{\tt py\_call(Obj:Func,Ret,Opts)}

\noindent
emulates {\tt py\_dot(Obj,Func,Ret,Opts)}.  Options and Error cases
are the same as for {\tt py\_func/[3,4]} and {\tt py\_dot/[3,4]}.

\indourjnsitem{py\_free(+ObjRef)}{py\_free/1}
%
  In general when \janus{} bi-translates between Python objects and
  Prolog terms it performs a copy: this has the advantage that each
  system can perform its own memory management independently of the
  other.  The exception is when a reference to a Python object is
  passed to XSB.  In this case, Python must explicitly be told that
  the Python object can be reclaimed, and this is done through {\tt
    py\_free/1}.

  {\bf Error Cases}

  \bi
\item {\tt ObjRef} is not a Python object reference.  (Only a syntax
  check is performed, so no determination is made that {\tt ObjRef} is
  a {\em valid} Python object reference
  \bi
  \item {\tt type\_error}
  \ei
  \ei
  
\ourrepeatmoditem{py\_pp(+Stream,+Options)}{py\_pp/2}{py\_pp}
\ourrepeatmoditem{py\_pp(+Stream,+Term)}{py\_pp/2}{py\_pp}
\indourmoditem{py\_pp(Term)}{py\_pp/1}{py\_pp}
%
Pretty prints the Prolog translation of a \janus{} Python term.  By
default, the term is translated to Python and makes use of Python's
{\tt pprint.pformat()}, which produces a string that is then returned
to Prolog and written out.  If the option {\tt prolog\_pp(true)} is
given, the term is pretty printed directly in Prolog.  As an example

\begin{verbatim}
pydict([''(name,'Bob'),''(languages,['English','French','GERMAN'])]).
\end{verbatim}

\noindent
is pretty-printed as 
\begin{verbatim}
{
  name:'Bob',
  languages:[
   'English','
   'French',
   'GERMAN'
  ]
} 
\end{verbatim}

Such pretty printing can be useful for developing applications such as
with {\tt jns\_elastic}, the \janus{} Elasticsearch interface.

\indourjnsitem{py\_add\_lib\_dir(+Path)}{py\_add\_lib\_dir/1}
%
This convenience predicate allows the user to add a path to the Python
library directories in a manner similar to {\tt add\_lib\_dir/1},
which adds Prolog library directories.

\indourjnsitem{py\_lib\_dirs(?Path)}{py\_lib\_dirs/1}
%
This convenience predicate returns the current Python library
directories as a Prolog list.

\indourjnsitem{values(+Dict,+Path,?Val)}{values/3}
%
  Convenience predicate to obtain a value from a (possibly nested)
  Prolog dictionary.  The goal

  {\tt values(D,key1,V)}

\noindent
  is equivalent to the
  Python expression {\tt D[key1]} while

  {\tt values(D,[key1,key2,key3],V)}
v
\noindent
is equivalent to the Python expression

{\tt D[key1][key2][key3]}.

There are no error conditions associated with this predicate.

\indourrepeatjnsitem{keys(+Dict,?Keys)}{keys/2}
\indourrepeatjnsitem{key(+Dict,?Keys)}{key/2}
\indourjnsitem{items(+Dict,?Items)}{items/2}
%
Convenience predicates (for the inveterate Python programmer) to
obtain a list of keys or items from a Prolog dictionary.  There are no
error conditions associated with these predicates.

The predicate {\tt key/2} returns each key of a dictionary on
backtracking, rather than returning all keys as one list, as in {\tt keys/2}.

\indourmoditem{obj\_dict(+ObjRef,-Dict)}{obj\_dict/2}{janus}
%
Given a reference to a Python object as {\tt ObjRef}, this predicate
returns the dictionary of attributes of {\tt ObjRef} in {\tt Dict}.
If no {\tt \_\_dict\_\_} attribute is associated with {\tt ObjRef} the
predicate fails.

{\tt obj\_dict/2} is a convenience predicate, and could be written
using {\tt py\_dot/4} as:

\begin{verbatim}
  py_dot('__main__',Obj,'__dict__',Dict).
\end{verbatim}

\indourmoditem{obj\_dir(+ObjRef,-Dir)}{obj\_dir/2}{janus}
%
Given a reference to a Python object as {\tt ObjRef}, this predicate
returns the list of attributes of {\tt ObjRef} in {\tt Dir}.  If no
\_\_dir\_\_ attribute is associated with {\tt ObjRef} the predicate
fails.

{\tt obj\_dir/2} is a convenience predicate, and could be written
using {\tt py\_dot/4} as:

\begin{verbatim}
  py_dot('__main__',Obj,'__dir__'(),Dir).
\end{verbatim}

\end{description}

\section{Performance and Space Management} \label{sec:jns-perf}

The core \janus{} routines -- {\tt py\_func/[3,4]} and {\tt py\_dot/4}
-- are written almost entirely in C, have shown good performance so
far, and continue to be optimized.  Calling a simple Python function
to increment a number from XSB and then returning the incremented
value to XSB should take about a microsecond or less on a reasonably
fast machine.  Of course, the overhead for passing large terms from
and to Python will be somewhat higher.  For instance, the time to pass
a list of integers from Python to XSB has been timed at about 20
nanoseconds per list element.  Nonetheless, for nearly any practical
application the time to perform useful functionality within Python
will far outweigh any \janus{} overhead.

Apart from system resource limitations, there is virtually no upper
limit on the size of Python structures passed back to Prolog: stress
tests have passed lists of integers of length 100 million from Python
to Prolog without problems. However, it should be noted that \janus{}t
must ensure that XSB's heap is properly expanded before copying a
large structure from Python to XSB, a topic to which we now turn.

\subsection{Memory Management} \label{sec:janus-memory}

\subsubsection{Space Management for XSB's Heap}
Because Python data structures are directly copied onto the XSB heap
stack, the heap serves as a buffer for the return of information from
Python.  XSB currently relies on the Python C API size routine {\tt
  Py\_SIZE} to estimate the size of a structure, since accessing this
routine has a constant-time overhead.\footnote{XSB estimates the size
  multiplying {\tt Py\_SIZE} by a constant factor.}  However, {\tt
  Py\_SIZE} only returns the length of a structure, and not its exact
size.  Accordingly long lists of large structures, heavily nested
dictionaries and other such structures may present a problem.  It
should be noted that when \janus{} is initialized, XSB's default
heap margin is reset to 1 megabyte so that any data structure whose
size is less than 1 megabyte will be copied safely, even if its size
is under-estimated.\footnote{XSB's default heap margin is 64 kbytes.}

Fortunately, this default works for most users.  In using \janus{}
the only times large data structures have proven a problem is when
returning large bulk queries from Elasticsearch, or returning large
sets of ontology instances from Wikidata.  In such a case there are
two options.  First, one may reset the heap margin to an even larger
value (see Volume 1 for details).  Alternately, one may use the option
{\tt sizecheck(true)} for those {\tt py\_func} or {\tt py\_dot} calls that
are expected to return large data structures.  If this option is
specified, the call performs two traversals of the data structure to
be returned: one to determine its size and ensure heap space, and another
to copy the data structure.

\subsection{Python Space Management}

When using Python's C interface, every Python object that is declared
in C must be explicitly released, a requirement that supports Python's
garbage collection.  \janus{} memory usage has been checked using
the {\tt guppy} package, which indicates that there are no memory
leaks.

%\subsection{Allowing Python to Reclaim Space}
%
%TBD: Discuss space issues for Python how they are now addressed and
%how they will be addressed in the future.

\section{Interfaces to Python Libraries}

The {\tt packages/janus/starters} directory contains code to interface
to various Python libraries---to help users start projects using
\janus{}.  Some of the files implement useful higher level mappings
that translate say, embedding spaces or SpaCy graphs to Prolog graphs,
or translate RDF graphs to lists of Prolog structures.  Others are
simple collections of examples to show how to query or update
Elasticsearch, to detect the language of input text or to perform
machine translation.  Nearly all of the interfaces have been a
starting point for research or commercial
applications.  \footnote{Testing has been done of the interfaces, but
the testing has not been exhaustive.  As a result, please double-check
any results, and report bugs -- or improvements -- to {\tt
  xsb.sorceforge.net}.}

When \janus{} is loaded, both the \janus{} directory and its
{\tt packages/janus/starters} sub-directory is added to the Prolog and
Python paths.  As a result, modules in these sub-directories can be
loaded into XSB and Python without changing their library paths.

Note that most of these applications require the underlying Python
libraries to have been installed via a {\tt pip} or {\tt conda}
install.

\subsection{Fasttext Queries and Language Detection: {\tt jns\_fasttext}}
%
Facebook's {\tt fastText} provides a collection of functionality that
includes querying pre-trained word vectors in over a hundred
languages~\cite{FBFJM18}, training sets of vectors, aligning vector
embeddings~\cite{MUSE2018}, and identifying languages via {\tt
  lid.176.bin}.  This XSB module uses the Python module {\tt fasttext}
and allows an XSB programmer to immediately start using fastText's
pre-trained word embeddings.  A related module, {\tt jns\_faiss}
provides an interface to Facebook's dense vector management system
Faiss.  The distinction between the two is that Faiss can manage
vectors read in from a file, and provides batch-oriented operations;
the fastText module relies on fastText's binary format and provides
simpler, though useful, query support.

\paragraph{Queries to Word Embeddings}
\begin{description}
  \indourmoditem{load\_model(+BinPath,+Name)}{load\_model/2}{jns\_fasttext}
  Loads a word embedding model in fastText binary form, the path of
  which is {\tt BinPath}.  {\tt Name} is an atom to be used as a
  Prolog referent, and so easily allows use of more than one word
  embedding model at a time.

  \indourmoditem{get\_nearest\_neighbors(+Name,+Word,-Neighbors)}{get\_nearest\_neighbors/3}{jns\_fasttext}
  Returns the 10 nearest neighbors of {\tt Word} in the model {\tt
    Name}.  This feature is useful for determining other words that
  are distributionally similar to {\tt Word}.  {\tt Neighbors} is a
  list of tuples (terms with functor {\tt ''/2}) containing a
  neighboring word and its cosine similarity to {\tt Word}.   Although
  {\tt Word} must be a Prolog atom, it need not be an actual English
  word.  Because fastText uses subword embeddings rather than word
  embeddings \cite{BGJM17}, {\tt Word} need not have been in the
  training set of the model.  This feature can sometimes be useful for
  correcting misspellings and other purposes.

  \indourmoditem{cosine\_similarity(+Name,+WordList,-SimMat)}{cosine\_similarity/3}{jns\_fasttext}
  For a model {\tt Name} and {\tt WordList} a list of atoms of length
  $N$, this predicate returns a (cosine) similarity matrix of
  dimension $N \times N$.
    
  \indourmoditem{get\_word\_vec(+Name,+Word,-Vec)}{get\_word\_vec/3}{jns\_fasttext}
  Returns a the vector for {\tt Word} in the model {\tt Name} as a
  Prolog list of floats.  In general, if a computation on word vectors
  can be done wholly on the Python side, it is much faster to do so,
  rather than manipulating vectors in XSB.  This is because the word
  vectors are actually kept as {\tt numpy} arrays and computations
  performed in C rather than in Python.
\end{description}

\paragraph{Language Identification via {\tt lid.176.bin}}
Assuming that Fasttext's language identification module is in the
current directory, the command:
\begin{verbatim}
 py_func(fasttext,load_model('./lid.176.bin'),Obj).
\end{verbatim}
Loads the model and unifies {\tt Obj} with a reference to the loaded
module which might look like {\tt pyObj(p0x7faca3428510)}.  
Next, a call to the example python module {\tt jns\_fasttext}:
\begin{verbatim}
py_func(jns_fasttext, fasttext_predict(pyObj(p0x7faca3428510),
       'janus is a really useful addition to XSB! But language detection
        requires a longer string than I usually want to type.'),Lang).  
\end{verbatim}
returns the detected language and confidence value, which in this case

\begin{verbatim}
''('__label__en',0.93856)
\end{verbatim}

Note that loading the model can be done by calling the Python {\tt
  fasttext} module directly.  In fact, the only reason that the module
{\tt jns\_fasttext} needs to be used (as opposed to calling the Python
functionality directly) is because the confidence of the language
detection is returned as a {\tt numpy} array, which \janus{} does
not currently translate automatically. \footnote{The examples {\tt
    jns\_fasttext} and {\tt googleTrans} were written by Albert Ki;
  {\tt jns\_faiss} was written by Albert Ki and Theresa Swift.}

\subsection{Dense Vector Queries with jns\_faiss}
The dense-vector query engine Faiss \cite{JDH17}, developed by
Facebook offers an efficient way to perform nearest neighbor searches
in vector spaces produced by word, network, tuple, or other
embeddings.  The {\tt jns\_faiss} example provides XSB predicates to
initialize a Faiss index from a text file of vectors, perform queries
to the index, and to make a weighted Prolog graph out of the vector
space.  

As with many machine-learning tools, Faiss expects that each of the
vectors is referenced by an integer.  For instance, a vector for the
string {\em cheugy} would be referenced by an integer, say 37.  The
XSB programmer thus would be responsible for associating the string
{\em cheugy} with 37 in order to use Faiss.  The main predicates
exported by {\tt jns\_faiss.P} include:

\begin{itemize}
\item {\tt faissInit(+XbFile,+Dim)} initializes a Faiss index where
  {\tt XbFile} is a text file containing the vectors to be indexed and
  {\tt Dim} is the dimension of these vectors. ({\tt xb} is Faiss
  terminology for the set of {\em base}, i.e., indexed, vectors.)
  This predicate also creates a {\tt numpy} array with a set of query
  vectors {\tt xq} consisting of the same vectors.  When the query and
  index vectors are set up in this manner, a nearest-neighbor search
  can be performed for any of the indexed vectors.  With this, the
  vector space can be explored, visualized, and so on.

  After execution of this predicate, a fact for the predicate {\tt
    jns\_faiss:xq\_num/1} contains the number of query vectors ({\tt
    xq}), which is the same as the number of indexed vectors ({\tt
    xb}).

\item {\tt get\_k\_nn(+Node,+K,-Neighbors)} finds the {\tt K} nearest
  neighbors of a node.  The predicate takes as input {\tt Node}, the
  integer identifier of a node, and {\tt K} the number of nearest
  neighbors to be returned.  The return structure {\tt Neighbors} is
  the Prolog representation of a 2-ary Python tuple (i.e., {\tt ''/2})
  containing as its first argument a list of {\tt K} distances and as
  it second argument a list of {\tt K} neighbors.


\item {\tt make\_vector\_graph(K)} Given a Faiss index, this predicate
  asserts a weighted graph in Prolog by obtaining the nearest {\tt K}
  neighbors for each indexed vector.  Edges of the graph have the form:

  {\tt vector\_edge\_raw(From,To,Dist)}

  \noindent
  Where {\tt From} and {\tt To} are integer referents for indexed
  vectors, and {\tt Dist} is the Euclidean distance between the vector
  with referent {\tt From} and the vector with referent {\tt To}.
  Each fact of {\tt vector\_edge\_raw/3} is indexed both on its first
  and second argument.

  If both the number of indexed vectors and {\tt K} are large,
  construction of the Prolog vector graph may take a few minutes.
  Construction time is almost wholly comprised of the time to find the
  set of {\tt K} nearest neighbors for each node.

\item {\tt vector\_edge(Node1,Node2,Dist)}.  The vector graph, which
  represents distances is undirected.  However to save space, the {\tt
    vector\_edge\_raw/3} facts are asserted so that if {\tt
    vector\_edge\_raw(Node1,Node2,Dist}) has been asserted, {\tt
    vector\_edge\_raw(Node2,Node1,Dist}) will not be asserted.  {\tt
    vector\_edge/3} calls {\tt vector\_edge\_raw/3} in both
  directions, and should be used for querying the vector graph.
  
\item {\tt write\_vector\_graph(+File,+Header)} writes out the vector
  graph to {\tt File}.  This predicate ensures that {\tt File}
  contains the proper indexing directive for {\tt vector\_edge\_raw/3}
  as well a directive to the compiler describing how to dynamically
  load {\tt File} in an efficient manner.  Because of these
  directives, the file can simply be consulted or ensure\_loaded and
  the user does not need to worry about which compiler options should
  be used.  The graph is loaded into the module {\tt vector\_graph}.

  {\tt Header} is simply a string that is written as a comment to the
  first line of {\tt File} that can serve to contain any necessary
  provenance information.
\end{itemize}  

\subsection{Translating Between RDF and Prolog: jns\_rdflib} \label{sec:jns-rdflib}
This module interfaces to the Python {\tt rdflib} library to read RDF
information from files in Turtle, N-triples and N-quads format, and to
write files in Turtle and N-triples format.  In addition, RDF HDT
files can be loaded and queried using rdflib-HDT.  As such {\tt
  jns\_rdflib} augments XSB's RDF package (Chapter~\ref{chapter:RDF})
which handles XML-RDF.

Within a triple, URIs and blank nodes are returned as Prolog atoms,
while literals are returned as terms with functor {\tt ''/3} (the
Prolog representation of a 3-ary tuple) in which the first argument is
the literal's string as a Prolog atom, the second argument is its
datatype, and the third argument its language. If the data type or
language are not included, the argument will be null.  As examples:

\begin{verbatim}
"That Seventies Show"^^<http://www.w3.org/2001/XMLSchema#string> 
\end{verbatim}
is returned as 
\begin{verbatim}
('"That Seventies Show"','<http://www.w3.org/2001/XMLSchema#string>',) 
\end{verbatim}
while 
\begin{verbatim}
"That Seventies Show"@en
\end{verbatim}
is returned as
\begin{verbatim}
('"That Seventies Show"',,en) 
\end{verbatim}

The file {\tt jns\_rdflib.P} contains predicates {\tt test\_nt/0}, {\tt
  test\_ttl/0}, {\tt test\_nq/0} to test reading and writing.  Note
that Python options needed to deserialize an RDFllb graph write are
specific to the RDFlib plug-in for a particular format, and these
plug-ins are not always consistent with one another.  As a result, if
other formats are desired, minor modifications of {\tt jns\_rdflib} may
be necessary, though they will often be simple to make.

The use of {\tt jns\_rdflib} differs on the RDF format used.  For the
{\tt turtle} (or {\tt ttl}), {\tt nt} (or {\tt ntriples}) and {\tt
  nquads} and {\tt jsonld} formats a file is read into XSB as a
(large) list, and an XSB list of terms of the proper form can be
transformed into RDF and written to a file.  For the {\tt HDT} format
the usage pattern is different: when an {\tt HDT} file is loaded, it
is simply memory mapped into a process and facts are loaded into a
rdflib graph (and into XSB) purely on demand~\footnote{At least I
think that's how it works...}.

\subsubsection{Functionality for Turtle, N-triples and N-quads Formats}

\paragraph{Reading RDF}
\begin{itemize}
\item {\tt read\_rdf(+File,+Format,-TripleList)} reads RDF from a file
  containing an RDF graph formatted as {\tt Format}, where the formats
  {\tt turtle}, {\tt nt}, {\tt jsonld} and {\tt nquads} have been
  tested.\footnote{The format {\tt ttl} is allowed as a substitute for
  {\tt turtle}, and {\tt ntriples} for {\tt nt}.}.  These formats can
  be tested on {\tt sample.ttl}, {\tt sample.nt} and {\tt sample.nq},
  all of which are in the {\tt packages/janus/starters} directory.

  Due to the structure of the Python {\tt RDFlib} graph, no guarantee
  is made that the order of facts in {\tt File} will match the order
  of facts in {\tt TripleList}.
\end{itemize}
  
{\bf Error Cases}
\bi
\item {\tt Format} is not {\tt nt}, {\tt turtle}, {\tt ttl}, or {\tt
  nquads} \bi
\item {\tt misc\_error}
\ei
\end{itemize}

\paragraph{Writing RDF}

If {\tt TripleList} is a list of terms, structured as {\tt ''/3} terms
described above, it can be easily be written to {\tt File} as properly
formatted RDF.  The Python function {\tt
  rdflib\_write\_file\_no\_decode()} can be called directly as:
\begin{verbatim}
py_func(jns_rdflib,rdflib_write_file_no_decode(+TripleList,+File),[format=+Fmt],-Ret).
\end{verbatim}
where {\tt Fmt} is {\tt turtle} or {\tt nq}.  {\tt
  rdflib\_write\_file\_no\_decode()} is a simple function that creates
an RDFlib graph out of {\tt TripleList}, serializes the graph and
prints it out.  The Python options needed to write to a file are
specific to the RDFlib plug-in for a particular format, so if other
formats are desired, minor modifications of {\tt jns\_rdflib} may be
necessary.

Due to the structure of the Python {\tt RDFlib} graph, no guarantee is
made that the order of facts in {\tt File} will match the order of
facts in {\tt TripleList}.
  
\subsubsection{Functionality for the HDT Format}

The RDF HDT format is intended to support large, read-only knowledge
bases, such as Wikidata, that may contain billions of triples.  A HDT
file is a compressed binary serialization that can be directly browsed
and queried.  The advantage of querying over compressed data is that
large data stores become manageable that otherwise wouldn't be.  For
instance, a Wikidata snapshot that contains several billion rows along
with indexes takes up about 160 Gbytes on disk and takes about 3
seconds to initialize into (jns\_)rdflib.  Furthermore, the data is
loaded into RAM only as needed for query evaluation.  

{\tt jns\_rdflib} offers two main predicates for use with rdflib HDT:

\begin{description}
  \indourmoditem{hdt\_load(+Store,-Obj)}{hdt\_load/2}{janus}
  %
  Initializes rdflib for the HDT file {\tt Store} and creates a rdflib
  graph in which to store the results of queries.  {\tt Obj} is the
  Python reference to the data store.
  
  \indourmoditem{hdt\_query(?Arg1,?Arg2,?Arg3,-List)}{hdt\_query/4}{janus}
  %

  Allows a user to query a HDT store using Prolog-like syntax and
  returning the results of the query in {\tt List}.  For instance the query
\begin{verbatim}
    hdt_query('http://www.wikidata.org/entity/Q144',Pred,Obj,List)
\end{verbatim}
finds all triples having the above URI as their subject.  In this
case, {\tt List} would be unified with a long list beginning with
\begin{footnotesize}
\begin{verbatim}
[('http://www.wikidata.org/entity/Q144','http://schema.org/name',''(dog,'',en))
 ('http://www.wikidata.org/entity/Q144','http://schema.org/description',''('domestic animal,'',en))
...
\end{verbatim}
\end{footnotesize}
\end{description}

\subsection{jns\_spacy}
SpaCy is widely used tool that exploits neural language models to
analyze text via dependency parses, named entity recognition, and much
else.  Although SpaCy is a Python tool, much of it is written in
C/Cython which makes it highly efficient.  The {\tt jns\_spacy} package
offers a flexible and efficient means to use SpaCy from Prolog (once
SpaCy has been properly installed for Python, along with appropriate
SpaCy language models).

In SpaCy, a user first loads one or more language models for the
language(s) of interest and of a size suitable to the application.
Text is then run through this language model and through other SpaCy
code producing a {\tt Document} object containing a great amount of
detail about the sentences in the text, tokens in the sentence and
their relations to one another.

Reflecting this sequence, the predicate {\tt load\_model/1} is used to
load a SpaCy model into the XSB/Python session:

\begin{verbatim}
load_model(en_core_web_sm) 
\end{verbatim}

\noindent
On the Python side the identifier {\tt en\_core\_web\_sm} is
associated with a {\tt Language} object, and using this
association the same atom can be used to process text throughout the
session.  Multiple models can be loaded and used to process different
text or files in different languages or for different purposes.  For
instance, the command:

\begin{verbatim}
proc_string(en_core_web_sm,'She was the youngest of the two daughters of a
most affectionate, indulgent father; and had, in consequence of her sister's
marriage, been mistress of his house from a very early period.',Doc)
\end{verbatim}

\noindent
processes the above text, and unifying {\tt Doc} with the referent to
the resulting SpaCy {\tt Document} object, which contains the textual
analysis of the string.  The predicate {\tt proc\_file/3} works
similarly for textual files.

At this point, a user of {\tt jns\_spacy} has two options: she can
either query the {\tt Document} object directly or call {\tt
  token\_assert/1} to assert information from the {\tt Document}
object into a Prolog graph that can be conveniently analyzed.  If
querying a {\tt Document} object directly, a small amount of code may
need to be written because SpaCy's API often returns generators to its
data structures rather than its data structures themselves.  Generally
the code that needs to be written is extremely simple and consists of
little more than a list comprehension: the functions {\tt get\_nps()},
{\tt get\_ents()} and {\tt get\_token\_info()} in {\tt jns\_spacy.py}
provide examples of this.

%The predicate {\tt load\_and\_proc/2} loads a SpaCy model and
%processes text, asserting Python references both to the document and
%to the function used to invoke the language model on text.  At this
%point, a user of \janus{} has two options: she can either query the
%document directly or call {\tt token\_assert} to assert information
%from the document into a Prolog graph that can be conveniently
%analyzed.  If querying a document directly, a small amount of code may
%need to be written because SpaCy's API often returns generators to its
%data structures rather than its data structures themselves.  Generally
%the code that needs to be written is extremely simple and consists of
%little more than a list comprehension: the functions {\tt get\_nps()},
%{\tt get\_ents()} and {\tt get\_token\_info()} in {\tt jns\_spacy.py}
%provide examples of this.

For most purposes however, it is easier to call the XSB predicate {\tt
  token\_assert(Doc)} that asserts tokens and their dependency parse
edges into XSB as explained below.  As an example of how to navigate
this graph, {\tt show\_all\_trees/0} and its supporting predicates
provide a simple but clear representation of the SpaCy dependency
parse in constituency tree form using the Prolog version of the parse.
Example~\ref{spacy-examp} below shows a similar sequence as it might
be executed in a simple session.


%The code in this package originated as part of a large research
%project and reflects the needs of that project.  Other applications
%may have slightly different needs, and the {\tt jns\_spacy} code can be
%easily refactored to support such needs.

As a final point before presenting the the main predicates, note that
if text from different languages is to be analyzed, the package {\tt
  jns\_fasttext} can be used to determine the language of a text
string, and the text can then be sent to one of several language
models. 

\paragraph{{\tt jns\_spacy} Predicates}

\begin{description}
  \ourrepeatmoditem{load\_model(+Model,+Options)}{load\_model/2}{jns\_spacy}
  \indourmoditem{load\_model(+Model)}{load\_model/1}{jns\_spacy}
%  
Loads the SpaCy model {\tt Model} and associates the Prolog atom {\tt
  Model} with the corresponding SpaCy {\tt Language} object.
Currently, the only form for {\tt Options} is a (possibly empty) list
of terms of the form {\tt pipe(Pipe)} where {\tt Pipe} is the name of
a SpaCy pipe, i.e., a process to add to the NLP pipeline of the SpaCy
Language object {\tt Model}.

{\tt load\_model(Model)} is a convenience predicate for {\tt
  load\_model(Model,[])}.

\ourrepeatmoditem{proc\_string(+Model,+Atom,-Doc,+Options)}{proc\_string/4}{jns\_spacy}
\indourmoditem{proc\_string(+Model,+Atom,-Doc)}{proc\_string/3}{jns\_spacy}
%
Processes the text {\tt Atom} using the model {\tt Model} and unifying
{\tt Doc} with the resulting SpaCy {\tt Document} object.  The only
option currently allowed in {\tt Options} is {\tt
  token\_assert}, which in addition asserts information from {\tt Doc}
into a Prolog graph (after removing information about any previous
dependency graphs).

{\tt proc\_string(+Model,+File,-Doc)} is a convenience predicate for\\
{\tt proc\_string(+Model,+Atom,-Doc,[])}.

If {\tt Model} has not been loaded, {\tt proc\_string/[3,4]} will try
to load it before processing.  If {\tt Model} cannot be found, a
Python {\tt NameError} error is thrown as an XSB miscellaneous error.

\ourrepeatmoditem{proc\_file(+Model,+File,-Doc,+Options)}{proc\_file/4}{jns\_spacy}
\indourmoditem{proc\_file(+Model,+File,-Doc)}{proc\_file/3}{jns\_spacy}
%
Opens {\tt File} and processes its contents using the model {\tt
  Model} and unifying {\tt Doc} with the resulting SpaCy {\tt
  Document} object. {\tt File} is opened in Python, and the stream for
{\tt File} is closed automatically.  The only option currently allowed
in {\tt Options} is {\tt token\_assert}, which in addition asserts
information from {\tt Doc} into a Prolog graph.

{\tt proc\_file(+Model,+File,-Doc)} is a convenience predicate for\\
{\tt proc\_file(+Model,+File,-Doc,[])}.

If {\tt Model} has not been loaded, {\tt proc\_file/[3,4]} will try to
load it before processing.  If {\tt Model} cannot be found, a Python
{\tt NameError} error is thrown as an XSB miscellaneous error.

\indourmoditem{token\_assert(+Doc)}{token\_assert/1}{jns\_spacy}
%    
This predicate accesses the SpaCy Document object {\tt Doc}, then
queries the dependency graph and other information from {\tt Doc}, and
asserts it to Prolog as a graph (after retracting information from any
previous dependency graphs).  The Prolog form of the graph uses two
predicates.  The first:
\begin{verbatim}  
    token_info_raw(Index,Text,Lemma,Pos,Tag,Dep,EntType)
\end{verbatim}
represents the nodes of the graph; For a given SpaCy {\tt token}
object the fields in the corresponding {\tt token\_info\_raw/7} fact are:
are as follows:
\begin{itemize}
  \item {\tt Index} ({\tt token.idx}) is the character offset of {\tt
    token} within the document (i.e., the input file or atom), and
    serves as an index for the token both in SpaCy and in its Prolog
    representation.
  \item {\tt Text} ({\tt token.text}) the verbatim form of {\tt token}
    in the text that was processed.
  \item {\tt Lemma} ({\tt token.lemma\_}) the base form of {\tt
    token}.  If {\tt token} is a verb, {\tt Lemma} is its stem, if
    {\tt token} is a noun, {\tt Lemma} is its singular form.
  \item {\tt Pos} ({\tt token.pos\_ }) is the coarse-grained part of
    speech for {\tt token} according to {\tt
      https://universaldependencies.org/docs/u/pos}
  \item {\tt Tag} ({\tt token.tag\_ }) The fine-grained part of speech
    for {\tt token} that contains some morphological analysis in
    addition to the part-of-speech.
    Cf.

    {\tt https://stackoverflow.com/questions/37611061/spacy-token-tag-full-list}

      for a discussion of its meaning and use.
  \item {\tt Dep} ({\tt token.dep\_ }) The type of relation that {\tt
    token} has with its parent in the dependency graph.
  \item {\tt EntType} ({\tt token.ent\_type\_ }) The SpaCy named entity
    type, e.g., person, organization, etc.
\end{itemize}  

Edges of the Prolog graph have the form:
\begin{verbatim}
token_childOf(ChildIndex,ParentIndex)
\end{verbatim}
where {\tt ChildIndex}, and {\tt ParentIndex} are indexes for {\tt
  token\_info\_raw/7} facts.

Note that SpaCy tokens have many other attributes, of which the above
are some of the more useful.  If other attributes are needed, the {\tt
  jns\_spacy} code can easily be expanded to include them.  However
many aspects of the parse can be easily reconstructed by the Prolog
graph and don't need to be materialized in Prolog.  For instance the
code for {\tt show\_all\_trees/0} in {\tt jns\_spacy.P} contains code
for constructing sentences, subtrees of a given token and so on.

\ourrepeatmoditem{get\_text(Index,Text)}{get\_text/2}{jns\_spacy}
\ourrepeatmoditem{get\_lemma(Index,Lemma)}{get\_lemma/2}{jns\_spacy}
\ourrepeatmoditem{get\_pos(Index,Pos)}{get\_pos/2}{jns\_spacy}
\ourrepeatmoditem{get\_tag(Index,Tag)}{get\_tag/2}{jns\_spacy}
\ourrepeatmoditem{get\_dep(Index,Dep)}{get\_dep/2}{jns\_spacy}
\ourrepeatmoditem{get\_ner\_type(Index,NER)}{get\_ner\_type/2}{jns\_spacy}
\indourmoditem{\tt token\_info(Index,Text,Lemma,Pos,Tag,Dep,Ent\_type)}{token\_info/7}{jns\_spacy}
%
Various convenience predicates for accessing {\tt token\_info\_raw/7}.
{\tt token\_info/7} is a convenience predicate that calls {\tt
  token\_info\_raw/7} and filters out spaces and punctuation.  {\tt
  get\_text/2}, {\tt get\_lemma/2} etc. get the appropriate field from
a {\tt token\_info\_raw/7} fact indexed by {\tt Index}.a
  
\indourmoditem{\tt show\_all\_trees()}{show\_all\_trees/0}{jns\_spacy}
%
  Given a SpaCy graph asserted to Prolog as described above, {\tt
    show\_all\_trees/0} navigates the graph, and for each sentence in
  the graph converts the dependency graph to a tree and prints it out.
  This predicate is useful for reviewing parses, and its code in {\tt
    jns\_spacy.P} can be modified and repurposed for other needed
  functionality.

\indourmoditem{sentence\_roots(-RootList)}{sentence\_root/1}{jns\_spacy}
%
Returns a list of the dependency graph nodes (i.e., {\tt
  token\_info\_raw/7} terms) that are roots of a sentence in the
Prolog dependency graph.  By backtracking through {\tt RootList},
sentence by sentence processing can be done for a document.

\indourmoditem{dependent\_tokens(+Root,-Toklist)}{dependent\_tokens/2}{jns\_spacy}
%
Given the index of token {\tt Root}, returns a sorted list of the
tokens dependent on {\tt Root}.  If {\tt Root} is the root of a
sentence, {\tt Toklist} will be the words in the sentence; if {\tt
  Root} is the root of a noun phrase, {\tt Toklist} will be the words
in the noun phrase, etc.

\end{description}  

\begin{example} \rm \label{spacy-examp}
We provide an example session where {\tt jns\_spacy} is used.  For a
session like this to work SpaCy would need to be installed along with
the SpaCy model {\tt en\_core\_web\_sm}.  The session would start by
consulting the appropriate files and model:
\begin{verbatim}
| ?- [janus,jns_spacy].
:
| ?- load_model(en_core_web_sm).
\end{verbatim}
Next SpaCy is used to process a string (i.e., a Prolog atom):
\begin{verbatim}
| ?- proc_string(en_core_web_sm,'She was the youngest of the two daughters of a most affectionate,indulgent father; and had, in consequence of her sister''''s marriage,been mistress of his house from a very early period.  ',Doc,[token_assert]).

Doc = pyObj(p0x7f36ed5b3580)

yes
\end{verbatim}
The option {\tt token\_assert} automatically loads the SpaCy
dependency graph and other information to Prolog.  Alternately, one
could omit this option and later call {\tt
  token\_assert(pyObj(p0x7f36ed5b3580))}: i.e., call {\tt
  token\_assert/1} with the first argument as the reference to the
SpaCy document object in Python.  Either way, once the dependency
graph has been loaded into XSB the command: 
\begin{verbatim}
show_all_trees().
\end{verbatim}
will print out the list of tokens for this sentence followed by:
    {\footnotesize
\begin{verbatim}
token_info(245,was,be,AUX,VBD,ROOT,)
   token_info(241,She,she,PRON,PRP,nsubj,)
   token_info(253,youngest,young,ADJ,JJS,attr,)
      token_info(249,the,the,DET,DT,det,)
      token_info(262,of,of,ADP,IN,prep,)
         token_info(273,daughters,daughter,NOUN,NNS,pobj,)
            token_info(265,the,the,DET,DT,det,)
            token_info(269,two,two,NUM,CD,nummod,CARDINAL)
            token_info(283,of,of,ADP,IN,prep,)
               token_info(317,father,father,NOUN,NN,pobj,)
                  token_info(286,a,a,DET,DT,det,)
                  token_info(293,affectionate,affectionate,ADJ,JJ,amod,)
                     token_info(288,most,most,ADV,RBS,advmod,)
                  token_info(307,indulgent,indulgent,ADJ,JJ,amod,)
   token_info(325,and,and,CCONJ,CC,cc,)
   token_info(375,been,be,VERB,VBN,conj,)
      token_info(329,had,have,AUX,VBD,aux,)
         token_info(334,in,in,ADP,IN,prep,)
            token_info(337,consequence,consequence,NOUN,NN,pobj,)
               token_info(349,of,of,ADP,IN,prep,)
                  token_info(365,marriage,marriage,NOUN,NN,pobj,)
                     token_info(356,sister,sister,NOUN,NN,poss,)
                        token_info(352,her,her,PRON,PRP$,poss,)
                        token_info(362,'s,'s,PART,POS,case,)
      token_info(380,mistress,mistress,NOUN,NN,attr,)
         token_info(389,of,of,ADP,IN,prep,)
            token_info(396,house,house,NOUN,NN,pobj,)
               token_info(392,his,his,PRON,PRP$,poss,)
      token_info(402,from,from,ADP,IN,prep,)
         token_info(420,period,period,NOUN,NN,pobj,)
            token_info(407,a,a,DET,DT,det,)
            token_info(414,early,early,ADJ,JJ,amod,)
               token_info(409,very,very,ADV,RB,advmod,)
\end{verbatim}
    }

A similar sequence, but with {\tt
  proc\_file(en\_core\_web\_sm,'emma.txt',Doc,[token\_assert])} parses
the sentences in {\tt emma.txt} and loads the results into XSB.  In
this case the command {\tt show\_all\_trees()} displays the dependency
graph for each sentence in tree form.
  \end{example}
 
\subsection{jns\_json}
This module contains an interface to the Python {\tt json} module,
with predicates to read JSON from and write JSON to files and strings.
The {\tt json} module transforms JSON objects into and from Python
dictionaries, which the interface maps to and from their term forms.
This module can be used to help understand how Python dictionaries
relate to XSB terms, or as an alternative to XSB's {\tt json} package
({\tt json} Chapter~\ref{chap:json}).  For instance, while for most
purposes XSB's {\tt json} package should be used, {\tt jns\_json} can
be useful if the json constructed and read comes from another \janus{}
application such as {\tt jns\_elastic}.  This is because the format
used by {\tt jns\_json} maps directly to a Python dictionary, while
that of the {\tt json} package maps to other (very useful) formats.

The {\tt jns\_json} functions are written in Python and can be
called directly from Prolog.
\begin{itemize}
\item {\tt py\_func(jns\_json,prolog\_load(+File),+Features,-Json)}
  opens and reads {\tt File} and returns its JSON content in {\tt
    Json} as a Prolog dictionary term.
\item {\tt py\_func(jns\_json,prolog\_dump(+Dict,+File),+Features,-Ret)}
  converts {\tt Dict} to a JSON object, write it to {\tt File} and
  returns the result of the operation in {\tt Ret}.
\item {\tt py\_func(jns\_json,prolog\_loads(+Atom),+Features,-Json)}
  reads the atom {\tt Atom} and returns its JSON content in {\tt Json}
  as a Prolog dictionary term.
\item {\tt  py\_func(jns\_json,prolog\_dumps(+Dict),+Features,-JsonAtom)}
  converts {\tt Dict} to a JSON string, and returns the string as the
  Prolog atom {\tt JsonAtom}.
\end{itemize}  

\subsection{Querying Wikidata from XSB}
%
{\sc Documentation is under construction, but the first versions of
  these modules are complete.}

Wikidata is a multi-language ontology-style knowledge graph created
from processing Wikipedia articles and from many other sources.  The
Wikidata graph contains a huge amount of information with 14-15
billion edges, This information consists of {\em Qnodes} which include
people, places, things and their classes.  Among the more important
Qnodes are of course, XSB ({\tt Q8042469}) and Prolog ({\tt Q163468}).
Qnodes are related to each other using {\em Pnodes}: among the more
important indicate that one node is a subclass of another ({\tt P279})
or an instance of another ({\tt P31}).  Both Pnodes and Qnodes have
various attributes such as their preferred label
(\verb|http://www.w3.org/2004/02/skos/core#prefLabel|).

Due to the amount of information it contains, Wikidata is widely used
in knowledge intensive applications such as NLP, entity resolution,
and content extraction along with many others.  However, Wikidata can
be difficult to use due to its size and its design.

In terms of its size, while Wikidata can be downloaded and stored in a
database, this can be time and resource intensive.  Alternately,
various Wikidata servers can be queried via REST interfaces, although
the public servers limit the number of queries made from a given
caller over a time period, making them useful only for a light query
load.  Easily usable snapshots of a Wikidata at a given time are also
available in RDF-HDT format (cf. Section
\ref{sec:jns-rdflib}).\footnote{Available at {\tt
  https://www.rdfhdt.org/datasets}.}  In using XSB with Wikidata, one
project found it worked well to query RDF-HDT first, with REST queries
as a backup.

The design of Wikidata also makes it difficult to use.  Information
useful to one project may simply be noise to another.  In addition
some Wikidata statements are reified, and others are not.  And
finally, the need to use identifiers such as {\tt P31} means that
aliases must be used for code readability.

XSB's Wikidata interfaces help address many of these issues. The HDT
interface {\tt jns\_wd} and the server interface {\tt jns\_wdi} were
both developed during a project that heavily used both XSB and
Wikidata.  While these interfaces worked well for our project, they
make no claim to tame all of Wikidata's difficulties, just the ones we
repeatedly ran into.

\subsubsection{{\tt jns\_wd}: Querying Wikidata via HDT}

\begin{description}
  \indourmoditem{wd\_query(?Arg1,?Arg2,?Arg3,?Lang)}{wd\_query/4}{jns\_wd}
  This predicate queries the HDT version of Wikidata and unifies the
  various arguments with Wikidata triples that match the input, so
  that the caller may backtrack through all results.  {\tt Arg1} and
  {\tt Arg3} can either be concise Qnode identifiers (e.g., {\tt
    Q144}: {\em dog}) or URLs that may or may not represent
  Qnodes.\footnote{Qnode identifiers are automatically expanded to
  URLs.}  Similarly, {\tt Arg2} may be a concise Pnode identifier
  (e.g., {\tt P31}) or a full URL. {\tt Lang} is a 2 character language
  designation, which serves as a filter if instantiated.  {\tt Arg3}
  can also be a string like {\tt Italy} which {\tt jns\_wd} turns into
  rdf form using Lang, e.g., {\tt Italy@en}.  This predicate is the
  basis of many other predicates in this module.

 In order to take advantage of HDT indexes, at least one of {\tt Arg1}
 and {\tt Arg3} should be instantiated; otherwise the query can take a
 long time.

Finally, there are many properties indicating provinance and other
meta-data that are not needed for many purposes.  The file {\tt
  jns\_wd\_ignore.P} defines the predicate {\tt ignore\_1/1} that
contains a number of Pnodes ({\tt Arg2} instantiations) that one
project preferred to filter out of the {\tt wd\_query/4} answers.
Filtering is off by default, and can be turned on by asserting {\tt
  jns\_wd:use\_wd\_filter}.  Of course, since filtering may be
application-specific, Pnodes can be added to or deleted from {\tt
  jns\_wd\_ignore.P} as desired.

\indourmoditem{wd\_get\_labels(+Qnode,-Label,?Lang)}{wd\_get\_labels/3}{jns\_wd}
Backtracks through all preferred labels
(\verb|http://www.w3.org/2004/02/skos/core#prefLabel|), and other
labels (\verb|http://www.w3.org/2004/02/skos/core#prefLabel| and \\
  \verb|http://www.w3.org/2000/01/rdf-schema#label|) whose language
  unifies with {\tt Lang}.

\indourmoditem{wd\_get\_label(+Qnode,-Label,?Lang)}{wd\_get\_label/3}{jns\_wd}
Tries to find a good label for {\tt Qnode} that unifies with {\tt
  Lang}, first trying for a preferred label, then
\verb|http://www.w3.org/2004/02/skos/core#prefLabel|, and finally
other labels (\verb|http://www.w3.org/2004/02/skos/core#prefLabel| and \\
\verb|http://www.w3.org/2000/01/rdf-szvchema#label|.

\ourrepeatmoditem{wd\_instance\_of(+SCNode,-CNode)}{wd\_instance\_of/2}{jns\_wd}
\ourrepeatmoditem{wd\_subclass\_of(+InstNode,-CNode)}{wd\_subclass\_of/2}{jns\_wd}
\indourmoditem{wd\_parent\_of(+Node,-Parent)}{wd\_parent\_of/2}{jns\_wd}
%
Because it is called with the first argument bound, {\tt
  wd\_subclass\_of/2} and {\tt wd\_instance\_of/2} both go up the
Wikidata ontology dag and should not have any problems with speed,
since upward traversals are supported by the HDT indexes.  These
predicate attempt to handle the case where the obtained class is a
reified statement.  In this case, it attempts another call from the
reified statement to try to get a Qnode, a strategy that works at
least {\em sometimes}.\footnote{Its for reasons like this that this
section is named {\tt starters} rather than {\tt
  perefctly\_finished\_interfaces} -- {\tt jns\_wd} cuts through some
of the brush, but not all of it.  And don't get me started on why some
of the classes are reified.}

In Wikidata, it is not always apparent whether a node has an instance
or a subclass relation with its parent, so {\tt wd\_parent\_of/3} is a
convenience predicate that calls both.

\end{description}

\subsubsection{{\tt jns\_wdi}: Querying Wikidata over the web}

This package provides a simple interface to the Wikidata website via
the Python library {\tt wikidataintegrator}.  It is one of two ways in
which \janus{} can be used to query Wikidata: the other is to query
a compressed local snapshot of Wikidata via the {\tt hdt}
functionality in {\tt jns\_rdflib}.  Each approach has advantages and
disadvantages.  The use of {\tt hdt} can be much faster: in part
because it requires no webservice calls, but also because the Wikidata
site slows down responses to requests from a session that is using the
site heavily.  On the other hand, to use {\tt hdt} the a Wikidata {\tt
  hdt} file must be locally mounted; and when a process loads the hdt
file, it must allocate a large amount of virtual memory, although this
memory does not usually affect RAM usage.\footnote{In Linux, this
  means that the process has a large virtual memory size, but its
  resident set size is low.}  So for applications that take place on
servers or that use Wikidata extensively the {\tt hdt} approach for
{\tt jns\_rdflib} is best; for other uses {\tt jns\_wdi} may be more
convenient.

\begin{description}
\indourmoditem{wdi\_get\_dict(+Qnode,-Dict))}{wdi\_get\_dict(+Qnode,-Dict)}{jns\_wdi}

\indourmoditem{wdi\_get\_entity(+Qnode,-EDict))}{wdi\_get\_entity/2}{jns\_wdi}

\indourmoditem{wdi\_sparql\_query(+Qnode,+PropertyNode,-Ret)}{wdi\_sparql\_query/3}{jns\_wdi}

\end{description}

\subsection{Other Interfaces, Examples and Demos}

\paragraph{jns\_elastic}
This module contains example code for using the Python {\tt
  elasticsearch} package.  A step by step description shows how a
connection is opened, and index is created and a document added and
committed.  The example then shows how the document can be searched in
two ways, and finally deleted.

Much of the information that Elasticsearch reads and writes is in JSON
format, which the Python interface transforms to dictionaries, and
\janus{} transforms these dictionaries to and from their term form.
Thus although this example is short, the ideas in it can easily be
extended to a full interface.\footnote{This has already been done by
one company that uses XSB.}  Often the {\tt elasticsearch} functions
can be called directly, but in certain cases simple Python functions
must be written to handle default positional
arguments. \footnote{\janus{} correctly handles default keyword
arguments, but the Python C API does not seem to support default
positional arguments.}


%\paragraph{mpl}
%This file has a simple example of how to create, display, and save as
%pdf a simple matplotlib document.

\paragraph{Reading XML files as \janus{} Dictionaries}
%
Although XSB's {\tt SGML} package allows XML files to be read, the
ability to read XML structures as \janus{} dictionaries can be
convenient, especially if an application already must navigate through
\janus{} dictionaries for other purposes.  The module {\tt
  jns\_xmldict}, based on the Python package {\tt
  xmltodict} \footnote{{\tt https://pypi.org/project/xmltodict}}
provides a simple implementation of this based on Python's {\tt Expat}
XML parser, and so retains the advantages of {\tt Expat} in terms of
reliability, Unicode support and speed.

\begin{description}
  \indourmoditem{xmldict\_read(+File,-Dict)}{xmldict\_read/2}{jns\_xmldict}
  
  Given an XML file {\tt File}, this predicate opens {\tt File},
  parses its contents, transforms the contents into a dictionary, and
  unifies {\tt Dict} with dictionary.  
\end{description}

\noindent
It is worthwhile noting that the Python {\tt xmltodict} package offers
several keyword arguments and other options for parsing XML files and
strings, that can be easily accessed via user-written \janus{} calls.

\paragraph{jns\_spellcheck}
This module provides a simple interface to {\tt pyspellchecker}, a
basic but sometimes useful spell checker and corrector based on
dictionaries and a minimum edit distance search.  Because a minimum
edit distance search is relatively expensive, it is best to check
whether a word is known via {\tt sp\_known/1}, and only call {\tt
  sp\_correct/2} on unknown words.

The two main predicates are:
\begin{description}
\indourmoditem{sp\_known(+Word)}{sp\_known/1}{jns\_spellcheck} Succeeds
  if {\tt Word} is known to the {\tt pyspellchecker} dictionary, and
  fails otherwise.

  \indourmoditem{sp\_correct(+WordIn,-WordOut)}{sp\_correct/2}{jns\_spellcheck}
  If {\tt Word} has a reasonable minimum-edit distance to a word {\em
    word$_1$} in the {\tt pyspellchecker} dictionary this predicate
  succeeds, unifying {\tt WordOut} to {\em word$_1$}; otherwise the
  predicate fails.
\end{description}


\paragraph{jns\_googleTrans}
This example provides demo code to access Google's web-services for
language translation and language detection using \janus .



\section{Known Limitations and Future Work}

\begin{itemize}
\item A callback mechanism is under development.  Currently it is
  possible for Python to call XSB and XSB to call back Python, but not
  for XSB to call Python, then Python to call back XSB.  
\item A possible future version may include a hook in XSB's atom
  garbage collection to list Python objects that may be garbage
  collected, and to send this information back to Python.  
\end{itemize}  

%%% Local Variables: 
%%% mode: latex
%%% TeX-master: "manual2"
%%% End: 
<
